%!TEX program = xelatex

\documentclass[lang=en, 11pt]{elegantpaper}

\newcommand{\ie}{\textit{i.e.}}
\newcommand{\eg}{\textit{e.g.}}

% Some self-cumstomed symbols which will be used in this article
\newcommand{\R}{\mathrm{R}}
\newcommand{\N}{\mathrm{N}}
\newcommand{\mC}{\mathrm{C}}
\newcommand{\F}{\mathrm{F}}
\newcommand{\V}{\mathrm{V}}
\newcommand{\M}{\mathcal{M}}

\newcommand{\Null}[1]{\text{null}~#1}
\newcommand{\range}[1]{\text{range}~#1}
\newcommand{\lmap}[2]{\mathcal{L}(#1, #2)}
\newcommand{\poly}[1]{\mathcal{P}(#1)}

\newcommand{\sol}{\textcolor{main}{\textbf{Solution}}}
\newcommand{\yes}{\textcolor{blue!60}{\textbf{$\checkmark$~YES}}}
\newcommand{\no}{\textcolor{red!80}{\textbf{$\times$~NO}}}
\newcommand{\property}{\itshape}

\newcommand{\norm}[1]{\left\lVert#1\right\lVert}
\newcommand{\mspan}[1]{\text{span}\left(#1\right)} % math: span operation in vectors
\newcommand{\avg}{\mathbf{avg}}
\newcommand{\rms}{\mathbf{rms}}
\newcommand{\sign}{\mathbf{sign}}
\newcommand{\std}{\mathbf{std}}
\newcommand{\diag}{\mathbf{diag}}
\newcommand{\tr}{\mathbf{tr}}
\newcommand{\one}{\mathbf{1}}


%\newcommand{\dfrac}{\displaystyle\frac}
\usepackage{tikz}
\usetikzlibrary{calc, shapes, positioning, arrows, through, angles, intersections, 3d}
\usepackage{titlesec}
%\titleformat{\paragraph}[hang]{\normalfont\normalsize}{\theparagraph.}{1em}{}
\usepackage[most]{tcolorbox}
\usepackage{fontspec}
\usepackage{amsmath}
\usepackage[nodisplayskipstretch]{setspace}
\usepackage{tocloft}

\linespread{1.35}
\setlist[1]{itemsep=5pt}
\renewcommand{\arraystretch}{0.74}
\renewcommand{\thefigure}{\arabic{section}.\arabic{figure}}
\numberwithin{equation}{section}

\newfontface\textfontface{Courier}[
    %UprightFont = Courier Regular,
    ItalicFont = Courier Oblique,
    BoldFont = Courier Bold,
    BoldItalicFont = Courier Bold Oblique
]


\newtcolorbox[auto counter, number within=section]{solution}[1][]{
    enhanced,
    breakable,
    fonttitle = \bfseries,
    colupper = blue!80,
    title = {Solution: },
    #1
}


\title{Solutions}
%\subtitle{Introduction to Applied Linear Algebra}
\author{utoppia}

\date{\today}
\version{1.0}

\cftsetindents{subsection}{0em}{4em}

\begin{document}
    \setmainfont{Courier}
    \pagenumbering{gobble}
    \begin{center}
        \vspace{10em}
        \huge \textbf{Solution for Exercises in}
    \end{center}
    \begin{flushright}
        \Large --- Linear Algebra Done Right
    \end{flushright}
    \vspace{30em}
    \begin{center}
        \large
        \begin{tabular}{l@{\hspace{1em}:\hspace{4em}}l}
            Author & utoppia  \\
            Last Update & \today \\
            Version & 1.0
        \end{tabular}
    \end{center}
    
    \clearpage

    %\input{secs/forewords.tex}
    \clearpage 

    \pagenumbering{roman}
    \tableofcontents
    \clearpage

    \pagenumbering{arabic}

    \section{Vector Space}

\subsection{$\R^n$ and $\mC^n$}

\begin{enumerate}
    \item Suppose $a$ and $b$ are real numbers, not both $0$. Find real numbers $c$ and $d$ such that 
        \[ 1/(a+bi) = c + di\] \label{ex:1A-1}
        \begin{solution}
            Since $c + di$ is the multiplicative inverse of $a + bi$, we have 
            \[ (a+bi) (c+di) = 1,\]
            expand the left-hand side of the equation above, we have 
            \[ (ac-bd) + (ad + bc) i = 1,\]
            \ie, 
            \[\begin{cases}
                ac - bd &= 1 \\
                ad + bc &= 0,
            \end{cases}\]
            since $a \neq 0$, from the below equation $ad + bc = 0$ we have $d = -bc/a$, substitute it in the above equation $ac-bd=0$, we have 
            \[ c = \dfrac{a}{a^2 + b^2},\]
            use $d = -bc / a$ we have 
            \[ d = -\dfrac{b}{a^2 + b^2}.\]
        \end{solution}
    \item Show that 
        \[ \dfrac{-1+\sqrt{3}i}{2}\]
        is a cube root of $1$ (meaning that its cube equals $1$).
        \begin{solution}
            We have 
            \begin{align*}
                \left(\dfrac{-1+\sqrt{3}i}{2}\right)^3 &= (\dfrac{-1+\sqrt{3}i}{2}\times \dfrac{-1+\sqrt{3}i}{2}) \times \dfrac{-1+\sqrt{3}i}{2} \\
                                &= \dfrac{(1-3) + (-\sqrt{3}-\sqrt{3})i}{4} \times \dfrac{-1+\sqrt{3}i}{2} \\
                                &= \dfrac{-1-\sqrt{3}i}{2} \times \dfrac{-1+\sqrt{3}i}{2} \\
                                &= \dfrac{(1+3)+(-\sqrt{3}+\sqrt{3})i}{4} \\
                                &= 1.
            \end{align*}
        \end{solution}
    \item Find two distinct square root of $i$.
        \begin{solution}
            One can be 
            \[ \dfrac{1+i}{2}\]
            since 
            \[ \dfrac{1+i}{2} \times \dfrac{1+i}{2} = \dfrac{(1-1)+(1+1)i}{2} = i .\]
            Another can be 
            \[  \dfrac{-1-i}{2} \]
            since 
            \[ \dfrac{-1-i}{2} \times \dfrac{-1-i}{2} = \dfrac{(1-1) + (1+1)i}{2} = i.\]
        \end{solution}
    \item Show that $\alpha+\beta = \beta + \alpha$ for all $\alpha,\beta \in \mC$.
        \begin{solution}
            Suppose $\alpha = a + bi$ and $\beta = c +di$, we have 
            \begin{align*}
                \alpha + \beta &= (a+bi) + (c+di) \\
                               &= (a+c) + (b+d)i \\
                               &= (c+a) + (d+b)i \\
                               &= (c+di) + (a+bi) \\
                               &= \beta + \alpha,
            \end{align*}
            where the second and forth equations above hold because of the definition of addition in $\mC$ and the third equation holds because 
            of the usual commutativity of addition in $\R$.
        \end{solution}
    \item Show that $(\alpha+\beta) + \lambda = \alpha + (\beta + \lambda)$ for all $\alpha, \beta, \lambda \in \mC$.
        \begin{solution}
            Suppose $\alpha = a+bi$, $\beta = c+di$, $\lambda = e + fi$, we have 
            \begin{align*}
                (\alpha + \beta) + \lambda &= \left((a+bi) + (c+di)\right) + (e + fi) \\
                                           &= \left( (a+c) + (c+b)i\right) + (e+fi) \\
                                           &= (a+c+e) + (b+d+f)i \\
                                           &= (a+(c+e)) + (b+(d+f))i \\
                                           &= (a+bi) + \left((c+e)+(d+f)i\right) \\
                                           &= \alpha + (\beta + \lambda),
            \end{align*}
            where second, third and fifth equations above hold because of the definition of the addition in $\mC$ and the fourth 
            equation holds because of the normal associativity in $\R$.
        \end{solution}
    \item Show that $(\alpha \beta) \lambda = \alpha(\beta \lambda)$ for all $\alpha, \beta, \lambda \in \mC$.
        \begin{solution}
            Suppose $\alpha = a+bi$, $\beta = c+di$ and $\lambda = e+fi$, we have 
            \begin{align*}
                (\alpha \beta) \lambda &= \left((a+bi)(c+di)\right)(e+fi) \\
                                       &= ((ac-bd)+(ad+bc)i)(e+fi) \\
                                       &= ((ac-bd)e-(ad+bc)f) + ((ac-bd)f + (ad+bc)e)i \\
                                       &= ((ce-df)a-(de+cf)b) + ((ce-df)b + (de+cf)a) i \\
                                       &= (a+bi)((ce-df)+(de+cf)i) \\
                                       &= (a+bi)((c+di)(e+fi)) \\
                                       &= \alpha (\beta \lambda),
            \end{align*}
            where the second, third, fifth and sixth equations above hold because of the definition of multiplicative in $\mC$ and 
            the fourth equation holds because of the normal form changing in $\R$.
        \end{solution}
    \item Show that for every $\alpha \in \mC$, there exists a unique $\beta \in \mC$ such that $\alpha + \beta = 0$.
        \begin{solution}
            Suppose $\alpha = a + bi$, it is easy to verify that $\beta = -a - bi$ satisfy the equaiton $\alpha + \beta = 0$, 
            now suppose there is another $\beta' \neq -a - bi$ which satisfy the equation, denote it as 
            \[ \beta' = c + di,\]
            we have 
            \[ \alpha + \beta' = (a+c) + (b+d)i = 0,\]
            implies that 
            \[ a + c = 0, \qquad b + d = 0,\]
            therefore 
            \[ c= -a, \qquad d=-b ,\]
            which is a contradictt with our assumption, therefore there exists a unique $\beta$ holds the equation.
        \end{solution}
    \item Show that for every $\alpha \in \mC$ with $\alpha \neq 0$, there exists a unique $\beta \in \mC$ such $\alpha \beta = 1$.
        \begin{solution}
            Suppose $\alpha = a + bi$, let $\beta = c + di$ satisfy the equation $\alpha \beta = 1$, implies
            \[ (ac-bd) + (ad+bc)i = 1, \]
            from the solution of exercise \ref{ex:1A-1}, we have 
            \[ c=\dfrac{a}{a^2 + b^2}, \qquad d=-\dfrac{b}{a^2+b^2}, \]
            and its unique.
        \end{solution}
    \item Show that $\lambda(\alpha + \beta) =\lambda \alpha + \lambda \beta $ for all $\alpha, \beta, \lambda \in \mC$.
        \begin{solution}
            Suppose $\alpha = a+bi$, $\beta = c + di$ and $\lambda = e + fi$, we have 
            \begin{align*}
                \lambda (\alpha + \beta) &= (e+fi)( (a+bi) + (c+di)) \\
                                         &= (e+fi)((a+c) + (b+d)i) \\
                                         &= (e(a+c)-f(b+d)) + (e(b+d)+f(a+c))i \\
                                         &= ((ea-fb) + (ec-fd)) + ((eb+fa) + (ed+fc))i\\
                                         &= ((ea-fb) + (eb+fa)i) + ((ec-fd)+(ed+fc)i)\\
                                         &= (e+fi)(a+bi) + (e+fi)(c+di) \\
                                         &= \lambda \alpha + \lambda \beta,
            \end{align*}
            where the second the fifth equations above hold because of the definition of addition in $\mC$ and the third and sisth equations hold 
            because of the definition of multiplicative in $\mC$, and the fourth equation holds in $\R$.
        \end{solution}
    \item Find $x \in \R^4$ such that 
        \[ (4,-3,1,7) + 2x = (5,9,-6,8).\]
        \begin{solution}
            Denote $x$ as 
            \[ x = (x_1, x_2, x_3, x_4), \]
            we have 
            \[ 2x = (2x_1, 2x_2, 2x_3, 2x_4)\]
            because of the definition of scalar multiplication in $\F^n$, therefore we have 
            \[ (4+2x_1, -3+2x_2, 1+2x_3, 7+2x_4) = (5,9,-6,8),\]
            which is same as 
            \[ 
                \begin{cases}
                    4 + 2x_1 &= 5\\
                    -3 + 2x_2 &= 9 \\
                    1 + 2x_3 &= -6 \\
                    7 + 2x_4 &= 8 
                \end{cases}
            \]
            solve the equation above we have 
            \[ x_1 =0.5, \qquad x_2 = 6, \qquad x_3=-3.5, \qquad x_4=0.5 ,\]
            thus 
            \[ x = (0.5, 6, -3.5, 0.5) .\]
        \end{solution}
    \item Explain why there does not exist $\lambda \in \mC$ such that 
        \[\lambda(2-3i,5+4i, -6+7i) = (12-5i, 7+22i, -31-9i).\]
        \begin{solution}
            This problem is same with finding $\lambda \in \mC$ which satisfies
            \[
                \begin{cases}
                    \lambda (2-3i) &= 12-5i \\
                    \lambda (5+4i) &= 7+22i \\
                    \lambda (-6+7i) &= -31-9i
                \end{cases},
            \]
            let $\lambda = a + bi$, where $a,b\in \R$, from the first equation $\lambda(2-3i)=12-5i$ and $(a+bi)(2-3i) = (2a+3b) + (2b-3a)i$ we have 
            \[ 2a + 3b = 12, \qquad 2b-3a = -5,\]
            which have solution
            \[ a = 3, \qquad b = 2,\]
            but $\lambda =3 +  2i $ does not satisfies the third equation, since 
            \[ (3 + 2i) (-6+7i) = -32 + 9i \neq -31-9i.\]
        \end{solution}
    \item Show that $(x+y)+z = x+(y+z)$ for all $x, y, z \in \F^n$.
        \begin{solution}
            \begin{align*}
                (x+y) + z &= \big((x_1, \ldots, x_n) + (y_1,\ldots, y_n)\big) + (z_1, \ldots, z_n) \\
                          &= (x_1+y_1, \ldots, x_n+y_n) + (z_1, \ldots, z_n) \\
                          &= (x_1+y_1+z_1, \ldots, x_n+y_n+z_n) \\
                          &= \big(x_1+(y_1+z_1), \ldots, x_n + (y_n+z_n)\big) \\
                          &= (x_1,\ldots,x_n) + (y_1+z_1, \ldots, y_n+z_n) \\
                          &= x + (y+z),
            \end{align*}
            where the second, third, fifth and sixth equations hold because of the definition of addition in $\F^n$, and the fourth equaiton holds because 
            of the associativity in $\F$.
        \end{solution}
    \item Show that $(ab)x = a(bx)$ for all $x\in \F^n$ and all $a, b \in \F$.
        \begin{solution}
            Let $x = (x_1, \ldots, x_n)$, we have 
            \begin{align*}
                (ab)x &= (ab)(x_1, \ldots, x_n) \\
                      &= (abx_1, \ldots, abx_n) \\
                      &= \big(a(bx_1), \ldots, a(bx_n)\big)\\
                      &= a(bx_1, \ldots, bx_n)
                      &= a(bx),
            \end{align*}
            where the second and fourth equations above hold because of the definition of the scalar-multiplication in $\F^n$,
            and the third equation holds because of the associativity in $\F$.
        \end{solution}
    \item Show that $1x = x$ for all $x \in \F^n$.
        \begin{solution}
            Let $x = (x_1, \ldots, x_n)$, we have 
            \begin{align*}
                1 x &= 1 (x_1, \ldots, x_n) \\
                    &= (1x_1, \ldots, 1x_n) \\
                    &= (x_1, \ldots, x_n) \\
                    &= x,
            \end{align*}
            where the second and fourth equations hold because of the definition of scalar-multiplication in $\F^n$, and 
            the third equation hold because of the identity in $\F$.
        \end{solution}
    \item Show that $\lambda(x+y) = \lambda x + \lambda y$ for all $\lambda \in \F$ and all $x,y \in \F^n$.
        \begin{solution}
            Let $x = (x_1, \ldots, x_n)$ and $y = (y_1, \ldots, y_n)$, we have 
            \begin{align*}
                \lambda (x + y) &= \lambda \big( (x_1,\ldots,x_n) + (y_1, \ldots, y_n)\big) \\
                                &= \lambda (x_1+y_1, \ldots, x_n+y_n) \\
                                &= \big(\lambda(x_1+y_1), \ldots, \lambda(x_n+y_n)\big) \\
                                &= (\lambda x_1 + \lambda y_1, \ldots, \lambda x_n + \lambda y_n) \\
                                &= (\lambda x_1, \ldots, \lambda x_n) + (\lambda y_1, \ldots, \lambda y_n) \\
                                &= \lambda (x_1, \ldots, x_n) + \lambda (y_1, \ldots, y_n) \\
                                &= \lambda x + \lambda y, 
            \end{align*}
            where the second and fifth equations hold because of definition of addition in $\F^n$, 
            the third and sixth equations hold because of the definition of scalar-multiplication in $\F^n$,
            the fourth equation holds because of the distributive property in $\F$.
        \end{solution}
    \item Show that $(a+b)x = ax + bx$ for all $a,b \in \F$ and all $x \in \F^n$.
        \begin{solution}
            Let $x = (x_1, \ldots, x_n)$ we have 
            \begin{align*}
                (a + b) x &= (a+b)(x_1, \ldots, x_n) \\
                          &= \big( (a+b)x_1, \ldots, (a+b)x_n\big) \\
                          &= (ax_1 + bx_1, \ldots, ax_n + bx_n) \\
                          &= (ax_1, \ldots, ax_n) + (bx_1, \ldots, bx_n) \\
                          &= a(x_1, \ldots, x_n) + b(x_1, \ldots, x_n) \\
                          &= ax + bx,
            \end{align*}
            where the second and fifth equations hold because of the definition of scalar-multiplication in $\F^n$, 
            and the fourth equation holds because of the definition of addition in $\F^n$,
            and the third equation holds because of the associativity and distributive property in $\F$, since 
            \[ (a+b)x_i = x_i(a+b) = x_ia + x_ib = ax_i + bx_i, \qquad i = 1, \ldots, n.\]
        \end{solution}
\end{enumerate}

\subsection{Definition of Vector Space}

\begin{enumerate}
    \item Prove that $-(-v) = v$ for every $v \in \V$.
        \begin{solution}
            Let $-(-v) = w$, since $w$ is the additive invers of $-v$, we have 
            \[ w + (-v) = 0,\]
            add by $v$ in the both sides of the equation above, we have 
            \[ w + (-v) + v = v,\]
            we can imply 
            \[ w = v ,\]
            since
            \[ (-v) + v = 0.\]
        \end{solution}
    \item Suppose $a \in \F, v \in \V$, and $av = 0$. Prove that $a=0$ or $v=0$.
        \begin{solution}
            If $a = 0$, then we solve the problem, therefore suppose $a \neq 0$, since $a \in \F$, it have multiplicative inverse, 
            denote it to be $a^{-1}$, \ie 
            \[ a a^{-1} = a^{-1}a =  1,\]
            from $av = 0$, we have 
            \[ a^{-1}av = a^{-1}0, \]
            from the left-hand side of the equation above we have 
            \[ a^{-1}av = (a^{-1}a)v = 1v = v,\]
            from the right-hand side of the equation above we have 
            \[ a^{-1}0 = 0,\]
            therefore we imply that 
            \[ v = 0,\]
            which is as desired.
        \end{solution}
    \item Suppose $v,w \in \V$. Explain why there exists a unique $x \in \V$ such that $v + 3x = w$.
        \begin{solution}  
            From $v + 3x = w$ we can imply that 
            \[ \dfrac{1}{3}(v-w) + x = 0,\]
            which means $x$ is the additive inverse of $\dfrac{1}{3}(v-w)$, thus $x$ is unique, since 
            \textbf{Every element in a vector space has a unique additive inverse}.
        \end{solution}
    \item The empty set is not a vector space. The empty set fails to satisfy only one of the requirements 
        listed in 1.19. Which one?
        \begin{solution}
            The \textbf{Additive identity}, since the additive identity requires at least one element $0$ exists in the vector space.
        \end{solution}
    \item Show that in the definition of a vector space (1.19), the additive inverse condition can be 
        replaced with the condition that 
        \[ 0v = 0 \text{ for all } v \in \V.\]
        Here the $0$ on the left side is the number $0$, and the $0$ on the right side is the additive identity 
        of $\V$. (The phrase "a condition can be replaced" in a definition means that the collection of 
        objects satisfying the definition is unchanged if the original condition is replaced with the 
        new condition.)
        \begin{solution}
            Suppose $v$ is a element in such set, We have 
            \begin{align*}
                v + 0 &= 1v + 0v \\
                      &= (1+0)v \\
                      &= 1v \\
                      &= v,
            \end{align*}
            where the second equaiton holds because of the distributive properties of this set, and the fourth equation holds 
            because of the multiplicative identity of this set. 
            
            Thus, the condition $0v = 0$ is same with $v + 0 = v$ if a set satisfies other conditions listed in 1.19.
        \end{solution}
    \item Let $\infty$ and $-\infty$ denote two distinct objects, neither of which is in $\R$. Define an 
        addition and scalar multiplication on $\R \cup \{\infty\} \cup \{-\infty\}$ as you could guess 
        from the notation. Specifically, the sum and product of two real number is as usual, and for 
        $t \in \R$ define 
        \[ t \infty = 
            \begin{cases}
                -\infty \quad& \text{if $t < 0$}, \\
                0 \quad& \text{if $t=0$}, \\
                \infty \quad& \text{if $t>0$},
            \end{cases} \qquad 
            t(-\infty) = 
            \begin{cases}
                \infty \quad& \text{if $t < 0$}, \\
                0 \quad& \text{if $t = 0$}, \\
                -\infty \quad& \text{if $t > 0$},
            \end{cases} \\
        \]
        \[
            t+\infty = \infty + t = \infty, \qquad t + (-\infty) = (-\infty) + t = -\infty, \\
        \]
        \[
            \infty + \infty = \infty, \qquad (-\infty) + (-\infty) = -\infty, \qquad \infty + (-\infty) = 0.
        \]
        Is $\R \cup \{\infty\} \cup \{-\infty\}$ a vector space over $\R$? Explain.
        \begin{solution}
            Yes. Because it satisfies all condition that the vector space should satisfy. For example let's check the commutative, 
            \ie, 
            \[ u + v = v + u, \qquad \text{for all $u,v \in \R \cup \{\infty\} \cup \{-\infty\}$},\]
            \begin{itemize}
                \item If $u, v \in \R$, the equation holds;
                \item otherwise, one of $u$ and $v$ should be $\infty$ or $-\infty$, since we have 
                    \[t+\infty = \infty + t = \infty, \qquad t + (-\infty) = (-\infty) + t = -\infty,\]
                    which means the equation holds.
            \end{itemize}

        \end{solution}
\end{enumerate}

\subsection{Subspaces}

\begin{enumerate}
    \item For each of the following subsets of $\F^3$, determine whether it is a subspace of $\F^3$;
        \begin{enumerate}[label=(\alph*)]
            \item $\{(x_1, x_2, x_3) \in \F^3: x_1 + 2x_2 + 3x_3 = 0\}$;
            \item $\{(x_1, x_2, x_3) \in \F^3: x_1 + 2x_2 + 3x_3 = 4\}$;
            \item $\{(x_1, x_2, x_3) \in \F^3: x_1x_2x_3=0\}$;
            \item $\{(x_1, x_2. x_3) \in \F^3: x_1 = 5x_3\}$.
        \end{enumerate}
        \begin{solution}
            For all solution below, we suppose $u$ and $v$ are the element of corresponded subsets. and $a$ to be a scalar, \ie, $a \in \F$,
            and denote $V$ to be the corresponded subsets.
            \begin{enumerate}[label=(\alph*)]
                \item $u + v = (u_1+v_1, u_2+v_2, u_3+v_3)$ and we have 
                    \[
                        (u_1+v_1) + 2(u_2+v_2) + 3(u_3+v_3) = (u_1+2u_2+3u_3) + (v_1+2v_2+3v_3) = 0 + 0 = 0,
                    \]
                    therefore 
                    \[ u + v \in V.\]
                    Also we have $au = (au_1, au_2, au_3)$ which holds the equation 
                    \[ au_1 + 2au_2 + 3au_3 = a(u_1+2u_2+eu_3) = 0,\]
                    therefore 
                    \[ au \in V ,\]
                    which menas $V$ is closed under addition and scalar multiplication, \ie, \textbf{it is a subspace}.
                \item Since $u + v = (u_1+v_1, u_2+v_2, u_3+v_3)$ and 
                    \[ (u_1+v_1) + 2(u_2+v_2) + 3(u_3+v_3) = (u_1+2u_2+3u_3) + (v_1+2v_2+3v_3) = 8 \neq 0,\]
                    which means that $V$ is not closed under addition, \ie, it is \textbf{not} a subspace.
                \item It's \textbf{not} a subspace, we can give a counterexample, it is easy to verify that 
                    \[ u=(1,1,0), \qquad v = (0,1,1) \]
                    are in $V$, but $u + v = (1,2,1)$ is not in $V$.
                \item For $u + v = (u_1+v_1, u_2+v_2, u_3+v_3)$ we have 
                    \[ (u_1 + v_1) = 5(u_3+v_3) ,\]
                    therefore 
                    \[ u + v \in V ,\]
                    and for $au = (au_1, au_2, au_3)$ we have 
                    \[ au_1 = 5au_3,\]
                    implies that 
                    \[ au \in V,\]
                    means that $V$ is closed under addition and scalar multiplication, \ie, it is a subspace.
            \end{enumerate}
        \end{solution}
    \item Verify all the assertions in Example 1.35.
        \begin{solution}
            \begin{enumerate}[label=(\alph*)]
                \item If the set 
                    \[ V = \{(x_1, x_2, x_3, x_4) \in \F^4: x_3 = 5x_4 + b\}\]
                    is a subspace, means it is closed under addition and scalar multiplication, \ie, 
                    for $u, v \in V$, we have $u + v \in V$, implies that 
                    \[ u_3 + v_3 = 5(u_4 + v_4) + b, \]
                    since we have $u, v \in V$, means 
                    \[ u_3 = 5u_4 + b, \qquad v_3 = 5v_4 + b,\]
                    we can imply that $b = 0$.

                    In another direction, if $b=0$, if $u, v \in V$, we have 
                    \[ u_3 + v_3 = 5(u_4 + v_4),\]
                    which means for $u + v in V$, 
                    and 
                    \[ au_3 = 5au_4,\]
                    which means that 
                    \[ au \in V,\]
                    therefore $V$ is a subspace.
                \item \textcolor{black}{The set of continuous real-valued functions on the interval $[0,1]$ is a subspace of $\R^{[0,1]}$.}
                    
                    Denote the set as $C^{[0,1]}$, since the function $0: [0,1] \to R$ defined as 
                    \[ 0(x) = 0 \]
                    is continuous, \ie, 
                    \[ 0 \in C^{[0,1]}.\]
                    
                    Suppose $f,g \in C^{[0,1]}$, since $f$ and $g$ are continuous real-value functions,
                    the $f + g$ is also a continuous real-valued function, \ie, 
                    \[ f + g \in C^{[0,1]},\] 
                    which means $C^{[0,1]}$ is closed under addtion.

                    let $\lambda \in \F$, the function $\lambda f$ is also a continuous function, means it exists in $C^{[0,1]}$, 
                    \ie, \[ \lambda f \in C^{[0,1]},\]
                    means that $C^{[0,1]}$ is closed under the scalar multiplication.

                    Finally, we conclude that $C^{[0,1]}$ is a subspace of $\R^{[0,1]}$.
                \item \textcolor{black}{The set of differentiable real-valued functions on $R$ is a subspace of $\R^{\R}$.}
                    
                    The function $0:\R \to \R$ defined as 
                        \[ 0(x) = 0\]
                    is differentiable, which means $0$ is in the set.

                    And for any functions $f, g$ in this set, since $f$ and $g$ are differentiable, implies that $f + g$ is differentiable, too,
                    \ie, $f + g$ is in this set.

                    For any scalar $\lambda$ and function $f$ in this set, we have $\lambda f$ is differentiable, \ie, $\lambda f$ is in this set.

                    Summary, we show that the set is not empty, and is closed under additon and scalar multiplication, \ie, it is a subspace.
                \item \textcolor{black}{The set of differentiable real-valued functions $f$ on the interval $(0,3)$ such that $f'(2)=b$ is a subspace of $\R^{(0,3)}$ if 
                    and only if $b=0$.}

                    If this set is a subspace, means that function $0$ defined as $0(x) = 0$ is in this set, and we have $0'(x)=0$, which implies that $b=0$.

                    Now suppose $b=0$, it is easy to verify that function $0$ is in this set,
                    and for functions $f$ and $g$ which are in this set, we have $f + g$ is differentiable, and 
                    \[(f+g)'(2) = f'(2) + g'(2) = 0\]
                    which means $f + g$ is in this set, too.

                    and for any scalar $\lambda$, the function $\lambda f$ is also differentiable, and we have 
                    \[ (\lambda f)'(2) = \lambda f'(2) = 0\]
                    which means that $\lambda f$ is in this set. 

                    Summary, we show that this set is not empty, and is closed under addition and scalar multiplication, we can conclude that this set is a subspace.
                \item \textcolor{black}{The set of all sequences of complex numbers with limit $0$ is a subspace of $\mC^{\infty}$.}
                
                    Since the sequences 
                    \[ 0, 0, 0, \ldots \]
                    has limit $0$, which means it is in this set. 

                    Suppose sequences $u$ and $v$ is in this set, we have 
                    \[ \lim_{k\to \infty} u_k = 0, \qquad \lim_{k \to \infty} v_k = 0,\]
                    therefore the limit of $u + v$ is 
                    \[ \lim_{k \to \infty} (u_k + v_k) = \lim_{k\to \infty} u_k + \lim_{k\to \infty} v_k = 0,\]
                    which means $u + v$ is in this set, \ie, this set is closed under addition.

                    For any scalar $a \in \mC$, we have 
                    \[ \lim_{k \to \infty} au_k = a\lim_{k \to \infty} u_k = 0,\]
                    which means $au$ is in this set, \ie, this set is closed under scalar multiplication.

                    Thus, the set is a subspace.
            \end{enumerate}
        \end{solution}
    \item Show that the set of differentiable real-valued functions $f$ on the interval $(-4,4)$ such 
        that $f'(-1)=3f(2)$ is a subspace of $R^{(-4,4)}$.
        \begin{solution}
            The function $0: (-4,4) \to \R$ define as 
            \[ 0(x) = 0, \]
            satisfies condition, means it's in this set.

            Suppose $f$ and $g$ are any functions in this set, which means they satisfies that 
            \[ f'(-1) = 3f(2), \qquad g'(-1) =3g(2),\]
            then we have 
            \[ (f+g)'(-1) = f'(-1) + g'(-1) = 3(f(2)+g(2)) = 3(f+g)(2),\]
            means $f + g$ is in this set.

            And with any $\lambda \in \F$, we have 
            \[ (\lambda f)'(-1) = \lambda f'(-1) = 3\lambda f(2) = 3(\lambda f)(2),\]
            means that $\lambda f$ is in this set.

            Thus, this set is a subspace.
        \end{solution}
    \item Suppose $b \in \R$. Show that the set of continuous real-valued functions $f$ on the interval 
        $[0,1]$ such that $\int_0^{1} f = b$ is a subspace of $\R^{[0,1]}$ if and only if $b=0$.
        \begin{solution}
            Suppose the set is a subspace, and $f$ and $g$ are any functions in this set, therefore $f+g$ is in this set, too, means 
            \[ \int_0^1 f = b,\qquad \int_0^1 g = b,\qquad \int_0^1 (f+g) = b,\]
            since 
            \[ \int_0^1 (f+g) = \int_0^1 f + \int_0^1 g = 2b,\]
            implies $2b = b$, \ie 
            \[ b = 0.\]
            
            In another direction, suppose $b =0$, \ie, the function in the set satisfies $\int_0^1 f = 0$.
            
            Obviously, the function $0: (-4,4) \to \R$ defined as 
            \[ 0(x) = 0\]
            satisfies the condition, therefore it is in this set.

            And for any functions $f$ and $g$ in this set, we have 
            \[ \int_0^1 (f+g) = \int_0^1 f + \int_0^1 g = 0 ,\]
            means $f + g$ is in this set, too.

            For any scalar $\lambda$, we have 
            \[ \int_0^1 (\lambda f) = \lambda \int_0^1 f = 0,\]
            means $\lambda f$ is in this set.

            Thus, we conclude that if $b=0$, the set described above is a subspace.
        \end{solution}
    \item Is $\R^2$ a subspace of the complex vector space $\mC^2$?
        \begin{solution}
            $\R^2$ is \textbf{not} a subspace of $\mC^2$, as we can give a counterexample: Suppose it is, 
            then we know $\R^2$ is closed under scalar multiplication, as $(1,0) \in \R^2$, let $\alpha = i$, 
            we have $i(1,0) = (i,0) \in \R^2$, which is contradictt.
        \end{solution}
    \item \begin{enumerate}[label=(\alph*)]
            \item Is $\{(a,b,c) \in \R^3: a^3=b^3\}$ a subspace of $\R^3$?
            \item Is $\{(a,b,c) \in \R^3: a^3  = b^3\}$ a subspace of $\mC^3$?
        \end{enumerate}
        \begin{solution}
            \begin{enumerate}[label=(\alph*)]
                \item Yes, since the additive identity is
                    \[ (0,0,0) ,\]
                    and it close under addition and scalar multiplication, since 
                    \[ (a+x, b+y, c+z) \in \R^3 \text{ and } (a+x)^3 = (b+y)^3 \]
                    if $(a,b,c), (x,y,z) \in \R^3$ and $a^3=b^3, x^3=y^3$, with same method, we can prove that 
                    it's closed satisfies the conditions.
                \item No, we can use the method of the previous problem.
            \end{enumerate}
        \end{solution}
    \item Give an example of a nonempty subset $U$ of $\R^2$ such that $U$ is closed under addition 
        and under taking additive inverse (meaning $-u \in U$ whenever $u \in U$), but $U$ is not 
        a subspace of $\R^2$.
        \begin{solution}
            Let $\N$ denote set of all the integer number, therefore 
            \[ \N^2 \subset \R^2 ,\]
            and 
            \[ (0,0) \in \N^2 \]
            is the additive identity, and it is closed under addition, since for $(a,b) \in \N^2$ and $(c,d) \in \N^2$, 
            we have 
            \[ (a+c, b+d) \in \N^2 ,\]
            meanwhile it is under taking additive inverse, for a element $u = (a,b) \in \N^2$, we have 
            \[ -u = (-a,-b) \in \N^2 ,\]
            but it is not closed under scalar multiplication, if we choos $a=0.5$ and $(1,1) \in \N^2$
            we have 
            \[ (0.5, 0.5) \notin \N^2 .\]
        \end{solution}
    \item Give an example of a nonempty subset $U$ of $\R^2$ such that $U$ is closed under scalar multiplication,
        but $U$ is not a subspace of $\R^2$.
        \begin{solution}
            Let the set of $U$ to be 
            \[ U = \{(x,y) \in \R^2: x^2 = y^2\},\]
            for any $u = (x,y) \in U$ and $a \in \R$, 
            we have $au = (ax, ay)$ which satisfies $(ax)^2 = (ay)^2$, \ie, 
            \[ au \in U ,\]
            but for $u=(1,-1), v=(1,1) \in U$, 
            we have 
            \[ u+v = (2,0) \notin U .\]
        \end{solution}
    \item A function $f: \R \to \R$ is called \textbf{periodic} if there exists a positive number $p$ such that 
        $f(x) = f(x+p)$ for all $x \in \R$. Is the set of periodic functions from $\R$ to $\R$
        a subspace of $\R^{\R}$? Explain.
        \begin{solution}
            Yes.

            First, we give a basic fact: if a function has periodic $p$, \ie, $f(x) = f(x+p)$, then we have 
            \[ f(x) = f(x + pk), \qquad k \in \N .\]

            Now, we can figure out that the additive identity is the periodic function $0:\R \to \R$ which is defined as
            \[ 0(x) = 0 .\]

            Suppose functions $f, g$ are element of this set, and we have 
            \[ f(x) = f(x+p_1), \qquad g(x) = g(x+p_2) ,\]
            let $p$ to be the lowest common multiple of $p_1$ and $p_2$, which means there exist $k_1, k_2 \in \N$, that 
            \[ k_1p_1 = k_2p_2 = p.\]
            according the fact we gived above, we have
            \[ f(x) = f(x+p_1) = f(x+p_1k_1) = f(x+p), \qquad g(x) = g(x+p_2) = g(x+p_2k_2) = g(x+p),\]
            therefore 
            \[ (f+g)(x) = f(x) + g(x) = f(x+p) + g(x+p) = (f+g)(x+p) ,\]
            which means that $f+g$ is a perodic function, \ie, this set is closed under addition.

            Suppose $f$ is in this set and it's periodic is $p$, and $\lambda \in \R$, we have 
            \[ (\lambda f)(x) = \lambda f(x) = \lambda f(x+p) = (\lambda f) (x+p) ,\]
            which means this set is closed under scalar multiplication.
        \end{solution}
    \item Suppose $U_1$ and $U_2$ are subspaces of $\V$. Prove that the intersection $U_1 \cap U_2$ is 
        a subspace of $\V$.
        \begin{solution}
            Since $U_1$ and $U_2$ are subspaces, the additive identity $0$ is in both set, which means 
            \[ 0 \in U_1 \cap U_2 .\]

            Now suppose $u, v \in U_1 \cap U_2$, \ie, $u,v \in U_1$ and $u,v \in U_2$,
            we have 
            \[ u+v \in U_1, \qquad u+v \in U_2\]
            since $U_1$ and $U_2$ are subspaces, therefore we have 
            \[ u+v \in U_1 \cap U_2 ,\]
            which means it is closed under addition.

            Suppose $u \in U_1 \cap U_2$, \ie, $u \in U_1$ and $u \in U_2$, and let $a \in \F$,
            we have 
            \[ au \in U_1, \qquad au \in U_2 \]
            since $U_1$ and $U_2$ are subspaces, therefore 
            \[ au \in U_1 \cap U_2 ,\]
            which means it is closed under scalar multiplication.

            Thus, $U_1 \cap U_2$ is a subspace of $\V$.
        \end{solution}
    \item Prove that the intersection of every collection of subspaces of $\V$ is a subspace of $\V$.
        \begin{solution}
            For convenient, Let $S_1, S_2, \ldots$ denote the subspaces of $\V$, and $I_1, I_2, \ldots$ 
            denote the intersection of the collections of subspaces. which means 
            \[ I_i = S_{k_1} \cap S_{k_2} \cap \dots .\]

            It is easy to verify that the additive identity $0$ is an element of $I_i$ since $0$ is an 
            element of every subspaces of $\V$. 

            Suppose $u, v \in I_i$, which means they are in every $S_{k_1}, S_{k_2}, \ldots$, since 
            $S_{k_1}, S_{k_2}, \ldots$ are subsapces, which means $u+v$ are in every  $S_{k_1}, S_{k_2}, \ldots$,
            \ie, 
            \[ u+v \in I_i ,\]
            which means $I_i$ is closed under addition.

            Suppose $a \in \F$, $u \in I_i$, implies that $u$ is in every $S_{k_1}, S_{k_2}, \ldots$, since $S_{k_1}, S_{k_2}, \ldots$
            are subspaces, we have $au$ in every $S_{k_1}, S_{k_2}, \ldots$, \ie 
            \[ au \in I_i ,\]
            which means $I_i$ is closed under scalar multiplication.
        \end{solution}
    \item Prove that the union of two subspaces of $\V$ is a subspace of $\V$ if and only if one of the 
        subspaces is contained in the other.
        \begin{solution}
            Denote the two subspaces of $\V$ as $U_1$ and $U_2$.

            First we suppose $U_1 \cup U_2$ is a subspace but neither $U_1$ nor $U_2$ is contained in the other, 
            which means there are elements $u$ and $v$ with 
            \[ u \in U_1 \text{ and }u \notin U_2,\qquad  v \in U_2 \text{ and } v \notin U_1 ,\]
            therefore 
            \[ u,v \in U_1 \cup U_2 ,\]
            with assumption $U_1 \cup U_2$ is a subpace, which means 
            \[ u+v \in U_1 \cup U_2 ,\]
            \ie, 
            \[ u+v \in U_1 \text{ or } u+v \in U_2.\]
            If $u + v \in U_1$ with $-u \in U_1$ we have $u+v+(-u) = v \in U_1$ which is contradictt.
            If $u + v \in U_2$ with $-v \in U_2$ we have $u+v+(-v) = u \in U_2$ which is contradictt, too.
            Therefore if $U_1 \cup U_2$ is a subspace, one of the $U_1$ and $U_2$ is contained in the other.

            In conversely direction, we can assume $U_1 \subseteq U_2$, implies that 
            \[ U_1 \cup U_2 = U_2 \]
            which is a subspace.
        \end{solution}
    \item Prove that the union of three subspaces of $\V$ is a subspace of $\V$ if and only if one of 
        the subspaces contains the other two.
        \begin{solution}
            Denote the three subspaces of $\V$ as $U_1, U_2, U_3$.


        \end{solution}
    \item Verify the assertion in Example 1.38.
        \begin{solution}
            For any $u \in U$ and $v \in W$, we have 
            \[ u+v = (x_1+x_2, x_1+x_2, y_1 + x_2, y_1 + y_2) \in \{(x,x,y,z)\in \F^4:x,y,z \in \F\},\]
            and for any $(x,x,y,z)$, we can find 
            \[ u=(x-y+z/2,x-y+z/2,z/2,z/2) \in U,\qquad v=(y-z/2,y-z/2,y-z/2,z/2)\in W,\]
            which satisfies
            \[ u + v = (x,x,y,z),\]
            as we desired.
        \end{solution}
    \item Suppose $U$ is a subspace of $\V$. What is $U + U$?
        \begin{solution}
            As $U$ is a subspace, it is closed under addtion, \ie, for any $u,v \in U$,
            we have $u + v \in U$, therefore 
            \[ U + U = U.\]
        \end{solution}
    \item Is the operation of addition on the subspaces of $\V$ commutative? In other words, 
        if $U$ and $W$ are subspaces of $V$, is $U + W = W + U$?
        \begin{solution}
            Suppose $x \in U + W$, according the definition, we have 
            \[ u \in U, \qquad v \in W, \]
            with 
            \[ x = u + v = v + u,\]
            which means $x \in W + U$. With same method, we can prove that if $x \in W + U$, we 
            have $x \in U + W$. 
            Thus, 
            \[ U+W = W+U.\]
        \end{solution}
    \item Is the operation of addition on the subspaces of $V$ associative? In other words, if 
        $U_1, U_2. U_3$ are subspaces of $V$, is 
        \[ (U_1 + U_2) + U_3 = U_1 + (U_2 + U_3)?\]
        \begin{solution}
            For any $u_1 \in U_1$, $u_2 \in U_2$, $u_3 \in U_3$, we have 
            \[ u_1 + u_2 + u_3 = (u_1+u_2) + u_3 \in (U_1 + U_2) + U_3,\]
            and 
            \[ u_1 + u_2 + u_3 = u_1 + (u_2 + u_3) \in U_1 | (U_2 + U_3),\]
            therefore 
            \[ (U_1 + U_2) + U_3 = U_1 + (U_2 + U_3) .\]
        \end{solution}
    \item Does the operation of addition on the subspaces of $V$ have an additive identity? Which subspaces have 
        additive inverses?
        \begin{solution}
            The subspace $0=\{0\}$ is an addition identity, since for any subspace $U$, we have 
            \[ U + 0 = U .\]

            Suppose $U$ has additive inverse $V$, \ie, 
            \[ U + V = 0,\]
            therefore for any $u \in U$, and $v \in V$, we have 
            \[ u + v = 0,\]
            since $U$ is a subspace, means for $-u \in U$, and 
            \[ -u + v = 0,\]
            we have 
            \[ v= 0,\]
            with same method, we have $-v \in V$ and $u + (-v) = 0$, which implies 
            \[ u=0,\]
            therefore $0$ is the only one subspace who has an additive inverse.
        \end{solution}
    \item Prove or give a counterexample: if $U_1, U_2$, $W$ are subspaces of $V$ such that 
        \[ U_1 + W = U_2 + W ,\]
        then $U_1 = U_2$.
        \begin{solution}
            We give a counterexample, let $V = \F^2$, and
            \[ W = \F^2,\qquad U_1=\{(x,0):x\in \F\},\qquad U_2=\{(0,x):x\in \F\},\]
            therefore $W,U_1,U_2$ are subspace of $V$, and 
            \[ U_1 + W = W = U_2 + W ,\]
            but 
            \[ U_1 \neq U_2 .\]
        \end{solution}
    \item Suppose 
        \[ U = \{(x,x,y,y) \in \F^4: x, y \in \F\}.\]
        Find a subspace $W$ of $\F^4$ such that $\F^4=U \oplus W$.
        \begin{solution}
            Let 
            \[ W = \{(x,y,y,0) \in \F^4: x,y,z \in F\},\]
            it is easy to verify that $W$ is a subspace, and 
            \[ U + W = \F^4 ,\]
            now we will show every element in $\F^4$ can be written in only one way as a sum of $u+w$:
            For $(x,y,z,w) \in \F^4$, we suppose $u = (p, p, q,q) \in U$ and $w = (m, n,n,0) \in W$ with 
            \[ u + w = (x,y,z,w), \]
            which is same as 
            \[
                \begin{cases}
                    p + m &= x \\
                    p + n &= y \\
                    q + n &= z \\
                    q &= w
                \end{cases},
            \]
            we have
            \[ p = y-z+w, \qquad q=w, \qquad m = x-y+z-w, \qquad n=z - w,\]
            which are unique as desired.
        \end{solution} 
    \item Suppose 
        \[ U = \{(x,y,x+y,x-y,2x) \in \F^5: x,y \in F\}.\]
        Find a subspace $W$ of $\F^5$ such that $\F^5 = U \oplus W$.
        \begin{solution}
            Let 
            \[ W = \{(0,0,x,y,z) \in \F^5: x,y,z \in \F\},\]
            $W$ is a subspace of $\F^5$, and 
            \[ U + W = \F^5 ,\]
            For any element $(x_1, x_2, x_3, x_4, x_5) \in \F^5$, the only way to format as sum of 
            elements of $U$ and $W$ is 
            \[ (x_1, x_2, x_1+x_2, x_1-x_2,2x_2) \in U, \qquad (0, 0, x_3-x_1-x_2, x_4-x_1+x_2, x_5-2x_1)\in W.\]
        \end{solution}
    \item Suppose 
        \[ U = \{(x,y,x+y,x-y,2x) \in \F^5: x,y \in \F\}.\]
        Find three subspaces $W_1,W_2,W_3$ of $\F^5$, none of which equals $\{0\}$, 
        such that $\F^5 = U \oplus W_1 \oplus W_2 \oplus W_3$.
        \begin{solution}
            We give $W_1, W_2, W_3$ as following:
            \begin{align*}
                W_1 &= \{(0,0,x,x,x) \in \F^5: x \in \F\}\\
                W_2 &= \{(0,0,0,x,x) \in \F^5: x \in \F\}\\
                W_3 &= \{(0,0,0,0,x) \in \F^5: x \in \F\}
            \end{align*}
            it is not difficult to verify that 
            \[ U \oplus W_1 \oplus W_2 \oplus W_3 = \F^5,\]
            since for any element $(x_1, x_2, x_3, x_4, x_5) \in \F^5$, we have 
            \begin{align*}
                u &= (x_1, x_2, x_1+x_2, x_1 - x_2, 2x_1) \in U \\
                w_1 &= (0,0,x_3-x_1-x_2,x_3-x_1-x_2,x_3-x_1-x_2) \in W_1 \\
                w_2 &= (0,0,0,x_4-x_3+2x_2, x_4-x_3+2x_2) \in W_2 \\
                w_3 &= (0,0,0,0,x_5-x_4-x_1-x_2) \in W_3.
            \end{align*}
        \end{solution}
    \item Prove or give a counterexample: if $U_1, U_2, W$ are subspaces of $V$ such that 
        \[ V = U_1 \oplus W \qquad \text{and} \qquad V = U_2 \oplus W,\]
        then $U_1 = U_2$.
        \begin{solution}
            I will give a counterexample, let $V = \R^2$, and 
            \[ W = \{(x,x) \in \R^2: x \in \R\},\]
            we have 
            \[ U_1 = \{(0,x)\in R^2:x\in\R\}, \qquad U_2 =\{(x,0)\in \R^2: x \in \R\},\]
            which satisfies
            \[ U_1 \oplus W = V = U_2 \oplus W,\]
            but 
            \[ U_1 \neq U_2.\]
        \end{solution}
    \item A function $f:\R \to \R$ is called \textbf{\textit{even}} if 
        \[ f(-x) = f(x) \]
        for all $x \in \R$. A function $f: \R \to \R$ is called \textbf{\textit{odd}} if 
        \[ f(-x) = -f(x) \]
        for all $x \in \R$. Let $U_e$ denote the set of real-valued even functions on $\R$
        and let $U_o$ denote the set of real-valued odd functions on $\R$. Show that $\R^{\R}=U_e \oplus U_o$.
        \begin{solution}
            It is easy to show that any real-valued function $f$ can be expressed as a sum of 
            an even function and an odd function, we let 
            \[ g(x) = \dfrac{f(x)+f(-x)}{2}, \qquad \phi(x) = \dfrac{f(x)-f(-x)}{2},\]
            we have $g(-x) = g(x)$ which means $g$ is even and $\phi(-x)=-\phi(x)$ means $\phi$ is odd, and 
            \[ f = g + \phi .\]

            The problem is that the form of $g$ and $\phi$ unique? Let suppose $g$ is an even function and $\phi$ is an odd function 
            with 
            \[ f(x) = g(x) + \phi(x) ,\]
            let $x = -x$, we have 
            \[ f(-x) = g(-x) + \phi(-x) = g(x) - \phi(x) ,\]
            from above two equations we have 
            \[ g(x) = \dfrac{f(x)+f(-x)}{2}, \qquad \phi(x) = \dfrac{f(x)-f(-x)}{2},\]
            as we desired.
        \end{solution}
\end{enumerate}
    \section{Finit-Dimensional Vector Spaces}

\subsection{Span and Linear Independence}

\begin{enumerate}
    \item Suppose $v_1, v_2, v_3, v_4$ spans $V$. Prove that the list 
        \[ v_1-v_2, v_2-v_3, v_3-v_4, v_4\]
        also spans $V$.
        \begin{solution}
            If $u \in \mspan{v_1, v_2, v_3,v_4}$, \ie 
            \begin{align*}
                u &= a_1v_1 + a_2v_2 + a_3v_3 + a_4v_4 \\
                  &= a_1(v_1-v_2) + (a_2+a_1)(v_2-v_3)+(a_3+a_2+a_1)(v_3-v_4) + (a_4+a_3+a_2+a_1)v_4,
            \end{align*}
            therefore $u \in \mspan{ v_1-v_2, v_2-v_3, v_3-v_4, v_4}$.

            Now suppose $u \in \mspan{ v_1-v_2, v_2-v_3, v_3-v_4, v_4}$, \ie 
            \begin{align*}
                u &= a_1(v_1-v_2) + a_2(v_2-v_3) + a_3(v_3-v_4) + a_4v_4 \\
                  &= a_1v_1 + (a_2-a_1)v_2 + (a_3-a_2)v_3 + (a_4-a_3)v_4
            \end{align*}
            therefore $u \in \mspan{v_1, v_2, v_3, v_4}$.

            Thus we have 
            \[ \mspan{v_1, v_2, v_3, v_4} = \mspan{ v_1-v_2, v_2-v_3, v_3-v_4, v_4}.\]
        \end{solution}
    \item Verify the assertions in Example 2.18.
        \begin{solution}
            \begin{enumerate}[label=(\alph*)]
                \item \textcolor{black}{A list $v$ of one vector $v \in V$ is linearly independent if and only if $v \neq 0$.}
                    First suppose $v$ is linearly independent and $v = 0$, for any $a \in \F$, we have $av=0$, therefore $v$ is linearly dependent, which is
                    a contradict.

                    If $v \neq 0$, the solution of equation $av = 0$, for $a \in \F$ is $a = 0$, therefore $v$ is linearly independent.
                \item \textcolor{black}{A list of two vectors in $V$ is linearly independent if and only if neither vector is a scalar multiple of the other.}
                    Suppose these two vectors are $u$ and $v$. If one is a scalar multiple of another, assume that $u = \lambda v$ for $\lambda \neq 0$,
                    we have 
                    \[ -u + \lambda v = 0\]
                    which means that $\{u,v\}$ is not linearly independent.

                    Now we assume that neither of $u$ and $v$ is a scalar multiple of the other and $\{u,v\}$ is linearly dependent, which menns there are nonzero 
                    $a_1, a_2$ with 
                    \[ a_1 u + a_2v = 0,\]
                    we can easy suppose that $a_1 \neq 0$, we have 
                    \[ u = -(a_2/a_1)v\]
                    which is a contradict to our assumption.
                \item \textcolor{black}{$(1,0,0,0),(0,1,0,0),(0,0,1,0)$ is linearly independent in $\F^4$.}
                    Let $a_1, a_2, a_3$ to be the coeffecients which make the sum of $(1,0,0,0),(0,1,0,0),(0,0,1,0)$ be zero, \ie, 
                    \[ a_1(1,0,0,0) + a_2(0,1,0,0)+a_3(0,0,1,0) =0,\]
                    which is 
                    \[ (a_1,a_2,a_3,0) = 0, \]
                    therefore 
                    \[ a_1 =0, \qquad a_2=0, \qquad a_3=0.\]
                \item \textcolor{black}{The list $1, z, \ldots, z^m$ is linearly independent in $\mathcal{P}(\F)$ for each nonnegative integer $m$.}
                    When $m=0$, this list is $1$ which is linearly independent, now suppose the list is linearly independent for $m=0,\ldots k$,
                    let 
                    \[ a_0 + a_1z + \dots + a_{k+1}z^{k+1} = 0,\]
                    let $z = 0$ we have $a_0=0$, the equation above is same with 
                    \[ a_1 z + \dots + a_{k+1}z^{k+1} = z(a_1 + \dots + a_{k+1}z^k) = 0,\]
                    which is same that 
                    \[ a_1 + \dots + a_{k+1}z^k = 0,\]
                    since $1, \ldots, z^{k}$ is linearly independent as we assumed, we have 
                    \[ a_1 = \dots = a_{k+1} = 0, \]
                    with 
                    \[ a_0 = 0, \]
                    we conclude that $1 ,\dots, z^{k+1}$ is linearly independent.
            \end{enumerate}
        \end{solution}
    \item Find a number $t$ such that 
        \[ (3,1,4), (2,-3,5), (5,9,t)\]
        is not linearly independent in $\R^3$.
        \begin{solution}
            Let $t = 2$, we have 
            \[ 3(3,1,4) + (-2)(2,-3,5) + (-1)(5,9,2) = (0,0,0),\]
            which means the list is not linearly independent.
        \end{solution}
    \item Verify the assertion in the second bullet point in Example 2.20.
        \begin{solution}
            \textcolor{black}{The list $(2,3,1), (1,-1,2), (7,3,c)$ is linearly dependent in $\F^3$ if and only if $c=8$, as you should verify.}

            If $c = 8$, with the first bullet point we know it is linearly dependent.
            
            Now suppose $c \neq 8$ and it is linearly dependent, 
            let $a_1,a_2,a_3$ be the coeffecients with 
            \[ a_1(2,3,1) + a_2(1,-1,2) + a_3(7,3,c) = 0,\]
            which is same as 
            \[
                \begin{cases}
                    2a_1 + a_2 + 7a_3 &= 0 \\
                    3a_1 - a_2 + 3a_3 &= 0 \\
                    a_1 + 2a_2 + ca_3 &= 0
                \end{cases},
            \]
            the first equation multiple by $7$ add the second equation multiple by $-3$ add the third equation multiple by $-1$, we 
            have 
            \[ (40-5c)a_3 = 0, \]
            since $c \neq 8$, we have $a_3 = 0$, substitue it to the equations above, we have $a_1 = a_2 = 0$, therefore 
            it's linearly independent, which is contradict.
        \end{solution}
    \item \begin{enumerate}[label=(\alph*)]
            \item Show that if we think of $\mC$ as a vector space over $\R$, then the list $(1+i, 1-i)$ is linearly independent.
            \item Show that if we think of $\mC$ as a vector space over $\mC$, then the list $(1+i, 1-i)$ is linearly dependent.
        \end{enumerate}
        \begin{solution}
            \begin{enumerate}[label=(\alph*)]
                \item Suppose $a_1, a_2 \in \R$ with 
                    \[ a_1 (1+i) + a_2 (1-i) = 0, \]
                    it is same with 
                    \[ 
                        \begin{cases}
                            a_1 + a_2 &= 0\\
                            a_1 - a_2 &= 0
                        \end{cases},
                    \]
                    which has only solution 
                    \[ a_1 = 0, \qquad a_2 = 0,\]
                    thus it is linearly independent.
                \item We can find $a_1 = 1+i, a_2 = 1-i \in \mC$, that 
                    \[ a_1 (1+i) + a_2(1-i) = 2i - 2i = 0,\]
                    therefore it is linearly dependent.
            \end{enumerate}
        \end{solution}
    \item Suppose $v_1, v_2, v_3, v_4$ is linearly independent in $V$. Prove that the list 
        \[ v_1-v_2, v_2-v_3,v_3-v_4,v_4\]
        is also linearly independent.
        \begin{solution}
            Assume that $ v_1-v_2, v_2-v_3,v_3-v_4,v_4$ is linearly dependent, there are nonzero $a_1, a_2, a_3, a_4$ with 
            \[ a_1 (v_1-v_2) + a_2(v_2-v_3) + a_3(v_3-v_4) + a_4v_4 = 0,\]
            which is same with
            \[ a_1v_1 + (a_2-a_1)v_2 + (a_3-a_2)v_3 + (a_4-a_3)v_4 = 0,\]
            since $v_1, v_2, v_3, v_4$ is linearly independent, therefore 
            \[ 
                \begin{cases}
                    a_1 &= 0 \\
                    a_2 - a_1 &= 0 \\
                    a_3 - a_2 &= 0 \\
                    a_4 - a_3 &= 0
                \end{cases},
            \]
            which means 
            \[ a_1 = a_2 = a_3 = a_4 = 0 ,\]
            which is a contradict to our assumption (that $a_1, a_2, a_3, a_4$ are nonzero).
        \end{solution}
    \item Prove or give a counterexample: If $v_1, v_2, \ldots, v_m$ is a linearly independent list of vectors in $V$, then 
        \[5v_1-4v_2, v_2,v_3,\ldots, v_m\]
        is linearly independent.
        \begin{solution}
            Assume that it is linearly dependent, therefore there are nonzero $a_1, \ldots, a_m$ that 
            \[ a_1(5v_1-4v_2) + a_2v_2 + \dots + a_mv_m = 0 ,\]
            which can be formed as 
            \[ 5a_1v_1 + (a_2-4a_1)v_2 + a_3v_3 + \dots + a_mv_m = 0,\]
            since $v_1, \ldots, v_m$ is linearly independent, we have 
            \[
                \begin{cases}
                    5a_1 &= 0 \\
                    a_2 - 4a_1 &= 0 \\
                    a_3 &= 0 \\
                    \vdots \\
                    a_m &= 0
                \end{cases},
            \]
            from the first equaiton $5a_1 = 0$ we have $a_1=0$, substitute it to the second equation $a_2-4a_1 = 0$, we have $a_2 = 0$, 
            therefore 
            \[ a_1 = a_2 = \dots = a_m = 0,\]
            which is a contradict to our assumption.

            Thus, this list is linearly independent.
        \end{solution}
    \item Prove or give a counterexample: If $v_1, v_2, \ldots, v_m$ is a linearly independent list of vectors in $v$ and $\lambda \in \F$ with 
        $\lambda \neq 0$, then $\lambda v_1, \lambda v_2, \ldots, \lambda v_m$ is linearly independent.
        \begin{solution}
            Suppose $a_1, \ldots, a_m \in \F$ that 
            \[ a_1\lambda v_1 + \dots + a_m \lambda v_m = 0,\]
            since $v_1, \ldots, v_m$ is linearly independent, means 
            \[a_i\lambda = 0, \qquad i = 1, \ldots, m,\]
            as $\lambda \neq 0$, we have 
            \[ a_i =0, \qquad i=1,\ldots,m,\]
            thus, the list is linearly independent.
        \end{solution}
    \item Prove or give a counterexample: If $v_1, v_2, \ldots, v_m$ and $w_1, \ldots, w_m$ are linearly independent lists of vectors in $\V$,
        then $v_1 + w_1, \ldots, v_m + w_m$ is linearly independent.
        \begin{solution}
            Let 
            \[ v_1 = (1,0), \qquad v_2 = (0,1), \qquad w_1=(0,1), \qquad w_2=(1,0),\]
            it is easy to verify that $v_1, v_2$ is linearly independent and $w_1, w_2$ is linearly independent,
            but $v_1+w_1, v_2+w_2$ which is 
            \[ (1,1), (1,1)\]
            is linearly dependent.
        \end{solution}
    \item Suppose $v_1, \ldots, v_m$ is linearly independent in $V$ and $W \in V$. Prove that if $v_1 + w, \ldots, v_m +w$ is linearly dependent, 
        then $w \in \mspan{v_1,\ldots,v_m}$.
        \begin{solution}
            Since $v_1 + w, \ldots, v_m + w$ are linearly dependent, which means there are nonzero $a_1, \ldots, a_m$ with 
            \[ a_1(v_1+w) + \dots + a_m(v_m + w)  = 0,\]
            let $S = a_1 + \dots + a_m$, the equation above is same with 
            \[ a_1v_1 + \dots + a_m v_m + Sw = 0,\]
            we have $S \neq 0$, or the equation above will be same with 
            \[ a_1 v_1 + \dots + a_mv_m = 0,\]
            and $v_1, \ldots v_m$ is linearly independent, implies 
            \[ a_1=\dots =a_m=0,\]
            which is a contradict with our assumption.
            Since $S \neq 0$, we can formed $w$ as 
            \[ w = \dfrac{-a_1}{S}v_1 + \dots + \dfrac{-a_m}{S}v_m ,\]
            \ie, 
            \[ w \in \mspan{v_1, \ldots, v_m}.\]
        \end{solution}
    \item Suppose $v_1, \ldots, v_m$ is linearly independent in $V$ and $w \in V$. Show that $v_1, \ldots, v_m, w$ is linearly independent if and only if 
        \[ w \notin \mspan{v_1, \ldots, v_m}.\]
        \begin{solution}
            First we suppose that $v_1, \ldots, v_m, w$ is linearly independent and $w \in \mspan{v_1, \ldots, v_m}$,
            therefore 
            \[ w = a_1 v_1 + \dots + a_m v_m ,\]
            \ie, 
            \[ a_1v_1 + dots + a_mv_m - w = 0,\]
            which means $v_1, \ldots, v_m, w$ is linearly dependent, which is contradict.

            Now we suppose $w \notin \mspan{v_1, \ldots, v_m}$ and $v_1, \ldots, v_m, w$ is linearly dependent, therefore there are 
            nonzero $a_1, \ldots, a_m, b$ that 
            \[ a_1 v_1 + \dots + a_mv_m + bw = 0,\]
            we have $b \neq 0$, or the equation above will be 
            \[ a_1v_1 + \dots + a_mv_m = 0,\]
            as $v_1, \ldots, v_m$ is linearly independent, implies that 
            \[ a_1 = \dots = a_m = 0,\]
            which is a contradict, since $b \neq 0$, we have 
            \[ w = \dfrac{-a_1}{b}v_1 + \dots + \dfrac{-a_m}{b}v_m,\]
            which means $w \in \mspan{v_1, \ldots, v_m}$, which is a contradict.
        \end{solution}
    \item Explain why no list of four polynomials spans $\mathcal{P}_4(\F)$.
        \begin{solution}
            We know that the list $1, z, z^2, z^3, z^4$ is linearly independent and they are in $\mathcal{P}_4(\F)$,
            with the fact that 
            \[ \text{\textit{Length of linearly independent list} $\leq$ \textit{Length of span list}}.\]
            We know the length of span list of $\mathcal{P}_4(\F)$ is at least $5$.
        \end{solution}
    \item \hypertarget{2A-13}{Prove} that $V$ is infinite-dimensional if and only if there is a sequence $v_1, v_2, \ldots$ of vectors in $V$ such that $v_1, \ldots, v_m$ is 
        linearly independent for every positive integer $m$.
        \begin{solution}
            Firts we will show that if $V$ is infinite-dimensional, then such a list exists: 
            we can use induction method in $m$,
            \begin{itemize}
                \item If $m=1$, there is non-zero elements in $V$, we choose $v_1 \neq 0 \in V$, therefore $v_1$ is linearly independent;
                \item Now suppose for $m=1,\ldots,k$, we can find $v_1, \ldots, v_k$ is linearly independent;
                \item For $m = k+1$, we can find $v_{k+1} \in V$ which can not be formed as linear combination of $v_1, \ldots, v_k$, 
                    or we have $\mspan{v_1, \ldots, v_k} = V$, which means $V$ is finit-dimensional, is contradict. Since $v_{k+1}$ 
                    can not be formed as linear combination of $v_1, \ldots, v_k$, means $v_1, \ldots, v_k, v_{k+1}$ is linearly independent.
            \end{itemize}

            Now we will show the inversely direction, that if such a list exist, that $V$ is infinite-dimensional:
            As fact it is easy, suppose that $V$ is finite-dimensional and $v_1, \ldots, v_k$ spans $V$, therefore there is 
            not linearly independent list with length greater than $k$, which is contradict to our assumption.
        \end{solution}
    \item Prove that $\F^{\infty}$ is infinite-dimensional.
        \begin{solution}
            
        \end{solution}
    \item Prove that the real vector space of all continuous real-valued functions on the interval $[0,1]$ is infinite-dimensional.
    \item Suppose $p_0,p_1,\ldots,p_m$ are polynomials in $\mathcal{P}_m(\F)$ such that $p_j(2)=0$ for each $j$.
        Prove that $p_0,p_1,\ldots,p_m$ is not linearly independent in $\mathcal{P}_m(\F)$.

\end{enumerate}

\subsection{Bases}

\begin{enumerate}
    \item Find all vector spaces that have exactly one basis.
        \begin{solution}
            The only vector space that has exactly one basis is $\{0\}$ and it is easy to verify. 

            Now suppose there is a vector space $U \neq \{0\}$ that has only one basis, assume the basis is 
            \[ v_1, \ldots, v_m .\]
            it is easy to verify that $v_i \neq 0$, 
            and we will show that 
            \[ v_1 + v_2, v_2, \ldots, v_m\] 
            is also a basis of U:
            let 
            \[ a_1 (v_1+v_2) + a_2v_2 + \dots + a_mv_m = 0,\]
            which is same with 
            \[ a_1v_1 + (a_1+a_2)v_2 + \dots + a_mv_m = 0,\]
            since $v_1, \ldots, v_m$ is basis therefore they are linearly independent, \ie, 
            \[ a_1 = \dots = a_m = 0,\]
            which means $v_1+v_2, v_1, \ldots, v_m$ is linearly independent.

            since $v_1, \ldots, v_m$ spans $U$, means for any $u \in U$, we have 
            \[ u = a_1v_1 + \dots + a_mv_m = a_1(v_1+v_2) + (a_2-a_1)v_2 + \dots + a_mv_m,\]
            means $v_1+v_2, v_2, \ldots, v_m$ spans $U$.

            Thus $v_1+v_2, v_2, \ldots, v_m$ is basis of $U$, but it is different with $v_1, \ldots, v_m$.
        \end{solution}
    \item Verify all the assertions in Example 2.28.
        \begin{solution}
            \begin{enumerate}[label=(\alph*)]
                \item \textcolor{black}{The list $(1,0,\ldots,0),(0,1,0,\ldots,0),(0,\ldots,0,1)$ is a basis of $\F^n$, called the \textbf{standard basis} of $\F^n$.}
                
                    For any $(x_1,\ldots, x_n) \in \F^n$, we can find 
                    \[ (x_1, \ldots, x_n) = x_1(1,0,\ldots) + \dots + x_n(0,\ldots,1),\]
                    therefore these list spans $\F^n$.

                    Suppose the we have 
                    \[ a_1 (1,0,\ldots,0) + \dots + a_n(0,\ldots,1) = (a_1, \ldots, a_n) = 0,\]
                    which means 
                    \[ a_1 = \ldots = a_n = 0,\]
                    therefore this list is linearly independent.
                \item \textcolor{black}{The list $(1,2),(3,5)$ is a basis of $\F^2$.}
                
                    For any $(x,y) \in \F$, we can find $a_1, a_2$ with 
                    \[ a_1(1,2) + a_2(3,5) = (x,y),\]
                    since the equations 
                    \[
                        \begin{cases}
                            a_1 + 3a_2 &= x \\
                            2a_1 + 5a_2 &=y 
                        \end{cases}
                    \]
                    has solution 
                    \[ a_1 = 3y-5x, \qquad a_2=2x-y.\]

                    Let $x=0,y=0$, we have 
                    \[ a_1 = a_2 = 0,\]
                    means this list is linearly independent.

                    Thus this list is a basis of $\F^2$.
                \item \textcolor{black}{The list $(1,2,-4), (7,-5,6)$ is linearly independent in $\F^3$ but is not a basis of $\F^3$ because it does not span $\F^3$.}
                
                    We can prove that this list is linearly independent since the equations
                    \[ 
                        \begin{cases}
                            a_1 + 7a_2 &= 0 \\
                            2a_1 - 5a_2 &= 0\\
                            -4a_1 + 6a_2 &= 0
                        \end{cases}
                    \]
                    has only solution 
                    \[ a_1 = a_2 = a_3 = 0.\]

                \item \textcolor{black}{The list $(1,1,0),(0,0,1)$ is a basis of $\{(x,x,y)\in \F^3: x,y \in \F\}$.}
                
                        It is easy to verigy that the list spans the vector space and it is linearly independent.s
                \item \textcolor{black}{The list $(1,-1,0),(1,0,-1)$ is a basis of \[\{(x,y,z)\in \F^3:x+y+z=0\}.\]}

                        It is easy to verify this list is linearly independent, now we will show that it spans the vector space:
                        since for any $(x,y,z)$ in this vector space, we can find 
                        \[ a_1 = -y,\qquad a_2 = -z\]
                        that 
                        \[ a_1(1,-1,0) + a_2(1,0,-1) = (x,y,z).\]
                \item \textcolor{black}{The list $1, z, \ldots, z^m$ is a basis of $\mathcal{P}_m(\F)$.}
                
                    We have prove that $1,z,\ldots,z^m$ is linearly independent, since it also spans $\mathcal{P}_m(\F)$, therefore it is a basis.
            \end{enumerate}
        \end{solution}
    \item \begin{enumerate}[label=(\alph*)]
            \item Let $U$ be the subspace of $\R^5$ defined by 
                \[ U = \{(x_1,x_2,x_3,x_4,x_5) \in \R^5:x_1=3x_2 \text{ and } x_3=7x_4\}.\]
                Find a basis of $U$.
            \item Extend the basis in part (a) to a basis of $\R^5$.
            \item Find a subspace $W$ of $\R^5$ such that $\R^5 = U \oplus W$.
        \end{enumerate}
        \begin{solution}
            \begin{enumerate}[label=(\alph*)]
                \item The list $(3,1,0,0,0),(0,0,7,1,0), (0,0,0,0,1)$ is a basis of this vector space.
                \item Since $(1,0,0,0,0)$ is not in the span of previous list, we can add $(1,0,0,0,0)$; then we find $(0,0,1,0,0)$ is not in the 
                    span of the previous list, therefore 
                    \[(3,1,0,0,0), (0,0,7,1,0),(0,0,0,0,1),(1,0,0,0,0),(0,0,1,0,0)\]
                    is a basis of $\R^5$.
                \item let $W = \mspan{(1,0,0,0,0), (0,0,1,0,0)}$, we have $U \oplus W = \R^5$.
            \end{enumerate}
        \end{solution}
    \item Prove or disprove: there exists a basis $p_0, p_1, p_2, p_3$ of $\mathcal{P}_3(\F)$ such that none of the polynomials $p_0, p_1, p_2, p_3$ has 
        degree $2$.
        \begin{solution}
            We will show that the polynomial list 
            \[ 1, x, x^3-x^2, x^3, x^4 \]
            satisfies the requirement, since none of the these polynomials has degree $2$, now we will show that it is a basis:
            since the list has length $5$, we only need to show that it is linearly independent: suppose that 
            \[ a_1 + a_2x + a_3(x^3-x^2) + a_4x^3 + a_5x^4 = 0,\]
            observed that only $a_5x^4$ has term $x^4$ implies $a_5 = 0$ also we have 
            \[ a_1 = a_2 =0,\]
            and only $a_3(x^3-x^2)$ has term $x^2$ implies that $a_2=0$, means $a_3 =0$, too, thus 
            it is linearly independent as desired.
        \end{solution}
    \item Suppose $v_1, v_2, v_3, v_4$ is a basis of $V$. Prove that 
        \[ v_1 + v_2, v_2+v_3, v_3+v_4, v_4\]
        is also a basis of $V$.
        \begin{solution}
            We have 
            \[ \dim{V} = 4,\]
            means we only need to show that $v_1 + v_2, v_2+v_3, v_3+v_4, v_4$ is linearly independent, then it is a basis of $V$ since it has the right length.
            suppose that 
            \[ a_1(v_1+v_2) + a_2(v_2 + v_3) + a_3(v_3+v_4) + a_4v_4 = 0,\]
            which can be formed as 
            \[ a_1v_1 + (a_1 + a_2)v_2 + (a_2 + a_3)v_3 + (a_3 + a_4)v_4 = 0,\]
            since $v_1, v_2, v_3, v_4$ is linearly independent, then we have 
            \[ a_1 = 0, \ a_1 + a_2 =0, \ a_2 + a_3 = 0, \ a_3 + a_4 = 0,\]
            which implies 
            \[ a_1 = a_2 = a_3 = a_4 = 0,\]
            as we desired.
        \end{solution}
    \item Prove or give a counterexample: If $v_1, v_2, v_3, v_4$ is a basis of $V$ and $U$ is a subspace of $V$ such that 
        $v_1,v_2 \in U$ and $v_3 \notin U$ and $v_4 \notin U$, then $v_1, v_2$ is a basis of $U$.
        \begin{solution}
            Consider $U = \mspan{v_1, v_2, v_3 + v_4}$, we will show that $U$ satisfies the conditions above, but $v_1,v_2$ is not the basis of $U$.

            It is obviousely that $v_1,v_2 \in U$, soppuse that $v_3 \in U$, we have $v_4 \in U$, means there is a linearly independent list with length $4$, 
            \ie, $v_1, v_2, v_3, v_4$ but the span list has length $3$, which is imposible, therefore 
            \[ v_3 \notin U,\]
            with same method, we have 
            \[ v_4 \notin U,\]
            thus $U$ satisfies the condition, and we know that the element $v_3 + v_4$ can not be formed as linear combination of $v_1$ and $v_2$, therefore 
            $v_1, v_2$ is not a basis of $U$.
        \end{solution}
    \item Suppose $U$ and $W$ are subspaces of $V$ such that $V = U \oplus W$. Suppose also that $u_1, \ldots, u_m$ is a basis of $U$ 
        and $w_1, \ldots, w_n$ is a basis of $W$. Prove that 
        \[ u_1, \ldots,u_m, w_1, \ldots, w_n\]
        is a basis of $V$.
        \begin{solution}
            First we will show that 
            \[ V = \mspan{u_1, \ldots, u_m, w_1, \ldots, w_n},\]
            since $V = U \oplus W$, we suppose that $p \in V$, therefore $p$ can be formed as 
            \[ p = u + w, \qquad u \in U, w \in W,\]
            and with $u_1, \ldots u_m$ is basis of $U$ and $u$ is an element of $U$, we have 
            \[ u = a_1 u_1 + \dots + a_mu_m,\]
            and simily we have 
            \[ w = b_1w_1 + \dots + b_nw_n,\]
            therefore 
            \[ p = a_1u_1 + \dots + a_mu_m + b_1w_1 + \dots + b_nw_n,\]
            means 
            \[ p \in \mspan{u_1, \ldots, u_m, w_1, \ldots, w_n}.\]
            Then we consider $p$ as an element of the span of list $u_1, \ldots, u_m, w_1, \ldots, w_n$, \ie, 
            \[ p = a_1u_1 + \dots + a_mu_m + b_1w_1 + \dots + b_nw_n ,\]
            since $u_1, \ldots, u_m$ is basis of $U$, we have 
            \[ a_1u_1 + \dots + a_mu_m \in U, \]
            let it to be $u$, \ie, 
            \[ u = a_1u_1 + \dots + a_mu_m \in U,\]
            similarly we have 
            \[ w = b_1w_1 + \dots + b_nw_n \in W,\]
            therefore 
            \[ p \in U + W = V,\]
            thus we have 
            \[ V = \mspan{u_1, \ldots, u_m, w_1, \ldots, w_n}.\]

            Since $U+W$ is a direct sum and therefore $u_1, \ldots, u_m, w_1, \ldots, w_n$ is linearly independent, implies that 
            \[ u_1, \ldots, u_m, w_1, \ldots, w_n\]
            is a basis of $V$.
        \end{solution}
\end{enumerate}

\subsection{Dimension}
\begin{enumerate}
    \item Suppose $V$ is finite-dimensional and $U$ is a subspace of $V$ such that $\dim{U} = \dim{V}$. Prove that $U = V$.
    \item Show that the subspaces of $\R^2$ are precisely $\{0\}$, $\R^2$, and all lines in $\R^2$ through the origin.
    \item Show that the subspaces of $\R^3$ are precisely $\{0\}$, $\R^3$, and all lines in $\R^3$ through the origin, and all
        planes in $\R^3$ through the origin.
    \item \begin{enumerate}[label=(\alph*)]
            \item Let $U = \{p \in \mathcal{P}_4(\F): p(6)=0.\}$. Find a basis of $U$.
            \item Extend the basis in part (a) to a basis of $\mathcal{P}_4(\F)$.
            \item Find a subspace $W$ of $\mathcal{P}_4(\F)$ such that $\mathcal{P}_4(\F) = U \oplus W$.
        \end{enumerate}
    \item \begin{enumerate}[label=(\alph*)]
            \item Let $U = \{p \in \mathcal{P}_4(\R):p''(6)=0\}$.Find a basis of $U$.
            \item Extend the basis in part (a) to a basis of $\mathcal{P}_4(\R)$.
            \item Find a subspace $W$ of $\mathcal{P}_4(\R)$ such that $\mathcal{P}_4(\R) = U \oplus W$.
        \end{enumerate}
    \item \begin{enumerate}[label=(\alph*)]
            \item Let $U = \{p \in \mathcal{P}_4(\R):p(2)=p(5)=p(6)\}$.Find a basis of $U$.
            \item Extend the basis in part (a) to a basis of $\mathcal{P}_4(\R)$.
            \item Find a subspace $W$ of $\mathcal{P}_4(\R)$ such that $\mathcal{P}_4(\R) = U \oplus W$.
        \end{enumerate}
    \item \begin{enumerate}[label=(\alph*)]
            \item Let $U = \{p \in \mathcal{P}_4(\R):\int_{-1}^{1} p =0\}$.Find a basis of $U$.
            \item Extend the basis in part (a) to a basis of $\mathcal{P}_4(\R)$.
            \item Find a subspace $W$ of $\mathcal{P}_4(\R)$ such that $\mathcal{P}_4(\R) = U \oplus W$.
        \end{enumerate}
    \item Suppose $v_1,\ldots,v_m$ is linearly independent in $V$ and $w \in V$. Prove that 
        \[ \dim{\mspan{v_1+w,\ldots,v_m+w}} \geq m - 1.\]
    \item Suppose $p_0,p_1,\ldots,p_m \in \mathcal{P}(\F)$ are such that each $p_j$ has degree $j$. Prove that 
        $p_0,p_1,\ldots,p_m$ is a basis of $\mathcal{P}_m(\F)$.
    \item Suppose that $U$ and $W$ are subspaces of $\R^8$ such that $\dim{U} = 3$, $\dim{W} = 5$, and 
        $U + W = \R^8$. Prove that $\R^8 = U \oplus W$.
    \item Suppose $U$ and $W$ are both five-dimensional subspaces of $\R^9$. Prove that $U \cap W \neq \{0\}$.
    \item Suppose $U$ and $W$ are both $4$-dimensional subspaces of $\mC^6$. Prove that there exist two vectors 
        in $U \cap W$ such that neigher of these vectors is a scalar multiple of the other.
    \item Suppose $U_1, \ldots, U_m$ are finite-dimensional subspaces of $V$. Prove that 
        $U_1 + \dots + U_m$ is finite-dimensional and 
        \[ \dim{(U_1 + \dots + U_m)} \leq \dim{U_1} + \dots + \dim{U_m}. \]
        \begin{solution}
            First We will show that \textbf{the sum of two finite-dimensional vector space is a finite-dimensional.}

            Suppose $U$ and $V$ are these two finite-dimensional vector space, therefore they have basis, since there are span list of them and
            span list include basis, suppose the that $u_1, \ldots, u_n$ and $v_1, \ldots, v_m$ are basis of $U$ and $V$, respectively. Now We will 
            show that 
            \[ u_1, \ldots, u_n, v_1, \ldots, v_m \]
            spans $U+V$: for any element $p \in U+V$, we have $p = u+v, u \in U, v \in V$, and we have 
            \[ u = a_1u_1 + \dots + a_nu_n, \qquad v = b_1v_1 + \dots + b_mv_m ,\]
            therefore 
            \[ p = a_1u_1 + \dots + a_nu_n + b_1v_1 + \dots + b_mv_m \]
            which is an element of $\mspan{u_1, \ldots, u_n, v_1, \ldots, v_m}$, and we can prove that for any element 
            in the $\mspan{u_1, \ldots, u_n, v_1, \ldots, v_m}$, it is also a element of $U + v$, therefore 
            \[ U+V = \mspan{u_1, \ldots, u_n, v_1, \ldots, v_m},\]
            \ie, $U +V$ is finite-dimensional.

            Use the inductive method, we can prove that $U_1 + \dots + U_m$ is finite-dimensional.

            In order to prove 
            \[ \dim{(U_1 + \dots + U_m)} \leq \dim{U_1} + \dots + \dim{U_m},\]
            we show that for any finite-dimensional vector space $U$ and $V$, we have 
            \[ \dim{(U+V)} = \dim{U} + \dim{V} - \dim{(U \cap V)} \leq \dim{U} + \dim{V}\]
            since $\dim{U \cap V} \geq 0$, therefore 
            \begin{align*}
                \dim{(U_1 + \dots + U_m)} &\leq \dim{U_1} + \dim{(U_2 + \dots + U_m)} \\
                                          &\leq \dots \\
                                          &\leq \dim{U_1} + \dots + \dim{U_m}.
            \end{align*}
        \end{solution}
    \item Suppose $V$ is finite-dimensional, with $\dim{V} = n \geq 1$. Prove that there exist $1$-dimensional subspaces $U_1, \ldots, U_n$ of 
        $V$ such that 
        \[ V = U_1 \oplus \dots \oplus U_n. \]
        \begin{solution}
            In order to show this, we prove another satetement: \textbf{For give $V$, we can find $1$-dimensional subsapce $U$ and $(n-1)$-dimensional $W$, that}
            \[ V = U \oplus W.\]
            We suppose that list 
            \[ v_1, \ldots, v_n\]
            is basis of $V$, let $U = \mspan{v_1}$ and $W = \mspan{v_2, \ldots,v_m}$, From 2.34 we know that 
            \[ V = U \oplus W,\]
            continue this method, we can prove that 
            \[ V = \mspan{v_1} + \dots + \mspan{v_n} .\]
        \end{solution}
    \item Suppose $U_1, \ldots, U_m$ are finite-dimensional subspaces of $V$ such that $U_1 + \dots + U_m$ is direct sum. 
        Prove that $U_1 \oplus \dots \oplus U_m$ is finite-dimensional and 
        \[ \dim{U_1 \oplus \dots \oplus U_m} = \dim{U_1} + \dots + \dim{U_m}. \]
        \begin{solution}
            If we can prove that when $m = 2$, the statement hold, then then statement hold for any positive interge $m$.

            Suppose that $u_1,\ldots,u_n$ and $v_1,\ldots, v_m$ are the basis of $U_1$ and $U_2$, respectively, then 
            \[ u_1, \ldots, u_n, v_1, \ldots, v_m\]
            spans $U_1 + U_2$, which means $U+V$ is finite-dimensional.
            
            Since $U_1 + U_2 $ is direct sum, we have $U_1 \cap U_2 = \{0\}$, \ie, $\dim{(U_1 \cap U_2)} = 0$ therefore 
            \[ \dim{(U_1 \oplus U_2)} = \dim{U_1} + \dim{U_2}.\]
        \end{solution}
    \item You might guess, by analogy with the formula for the number of elements in the union of three subsets of a finite set, 
        that if $U_1, U_2, U_3$ are subspaces of a finite-dimensional vector space, then 
        \begin{align*}
            \dim{(U_1 + U_2 + U_3)} =& \dim{U_1} + \dim{U_2} + \dim{U_3} \\
                                   & - \dim({U_1 \cap U_2)} - \dim{(U_1 \cap U_3)} - \dim{(U_2 \cap U_3)} \\
                                   & + \dim{(U_1 \cap U_2 \cap U3)}.
        \end{align*}
        Prove this or give a counterexample.
        \begin{solution}
            We will give a counterexample:
            Consider that 
            \[ U_1=\{(x,x)\in \R^2:x\in \R\},\ U_2=\{(x,0)\in \R^2: x\in \R\}, \ U_3=\{(0,x)\in \R^2: x\in \R\},\]
            we have 
            \[ \dim{U_1} = \dim{U_2} = \dim{U_3} = 1,\]
            and since $U_1 + U_2 + U_3 = \R^2$, we have 
            \[ \dim{(U_1 + U_2 + U_3)} = 2,\]
            and 
            \[ U_1 \cap U_2 = U_1 \cap U_3 = U_2 \cap U_3 = U_1 \cap U_2 \cap U_3 = \{0\},\]
            we have 
            \[ \dim{(U_1\cap U_2)} = \dim{(U_1 \cap U_3)} = \dim{(U_2 \cap U_3)} = \dim{(U_1 \cap U_2 \cap U_3)} = 0,\]
            therefore the left-hand side of the equation is $2$, the right-hand side is $2$, since 
            \[ 3 \neq 2,\]
            we conclude that the equation does not hold all the time.
        \end{solution}
\end{enumerate}
    \section{Linear Maps}

\subsection{The Vector Space of Linear Map}

\begin{enumerate}
    \item Suppose $b,c \in \R$. Define $T: \R^3 \to \R^3$ by 
        \[ T(x,y,z) = (2x-4y+3z+b, 6x+cxyz).\]
        Show that $T$ is linear if and only if $b=c=0$.
    \item Suppose $b,c \in \R$. Define $T: \mathcal{P}(\R) \to \R^2$ by 
        \[ Tp = \left(3p(4)+5p'(6)+bp(1)p(2), \int_{-1}^{2}x^3p(x)dx + c\sin p(0)\right).\]
        Show that $T$ is linear if and only if $b = c = 0$.
    \item Suppose $T \in \mathcal{L}(\F^n, \F^m)$. Show that there exist scalars $A_{j,k} \in \F$ for $j=1,\ldots,m$ and $k=1,\ldots,n$ such that 
        \[T(x_1,\ldots,x_n) = (A_{1,1}x_1+\dots+A_{1,n}x_n, \ldots, A_{m,1}x_1+\dots +A_{m,n}x_n)\]
        for every $(x_1,\ldots,x_n) \in \F^n$.
    \item Suppose $T \in \mathcal{L}(V,W)$ and $v_1, \ldots,v_m$ is a list of vectors in $V$ such that $Tv_1,\ldots,Tv_m$ is a linearly independent list 
        in $W$. Prove that $v_1, \ldots, v_m$ is linearly independent.
    \item Prove the assertion in 3.7.
    \item Prove the assertion in 3.9.  
    \item Show that every linear map from a $1$-dimensional vector space to itself is multiplication by some scalar. More precisely, prove that if $\dim{V}=1$
        and $T \in \mathcal{L}(V,V)$, then there exists $\lambda \in \F$ such that $Tv = \lambda v$ for all $v \in V$.
    \item Give an example of a function $\phi: \R^2 \to \R$ such that 
        \[ \phi(av) = a\phi(v)\]
        for all $a \in \R$ and all $v \in \R^2$ but $\phi$ is not linear.
    \item Give an example of a function $\phi: \mC \to \mC$ such that 
        \[ \phi(w+z) = \phi(w) + \phi(z)\]
        for all $w,z \in \mC$ but $\phi$ is not linear.
    \item Suppose $U$ is a subspace of $V$ with $U \neq V$. Suppose $S \in \mathcal{L}(U,W)$ and $S \neq 0$(which means that $Su \neq 0$ for some $u \in U$).
        Define $T: V \to W$ by 
        \[ Tv = 
            \begin{cases}
                Sv \qquad& \text{if $v \in U$}, \\
                0 \qquad& \text{if $v \in V$ and $v \notin U$.}
            \end{cases}
        \]
        Prove that $T$ is not a linear map on $V$.
    \item Suppose $V$ is finite-dimensional. Prove that every linear map on a subspace of $V$ can be extended to a linear map on $V$.
        In other words, show that if $U$ is a subspace of $V$ and $S \in \mathcal{L}(U,W)$, then there exists $T \in \mathcal{L}(V,W)$ such 
        that $Tu = Su$ for all $u \in U$.
        \begin{solution}
            Since $U$ is a subspace of $U$, let $u_1,\ldots,u_n$ to be the basis of $U$, and it can extended to be basis of $V$, suppose it is 
            \[ u_1, \ldots, u_n, w_1, \ldots, w_m,\]
            We define $T: V \to W$:
            \[ T(a_1u_1 + \dots + a_nv_n + b_1w_1 + \dots + b_mw_m) = S(a_1u_1+\dots+a_nu_n) + b_1w_1 + \dots + b_mw_m, \]
            where $a_1,\ldots,a_n,b_1,\ldots, b_m$ are arbitrary elements of $\F$, since $u_1,\ldots,u_n,w_1,\ldots,w_m$ is a basis of $V$, therefore 
            the equation define a function $T$ from $V$ to $V$. 

            Now for $u \in U$, we have 
            \[ u = c_1u_1 + \dots + c_nu_n, \]
            therefore 
            \[ Tu = Su .\]

            Now we will show that $T$ is linear map, for $u, v \in V$, suppose that 
            \[ u = a_1u_1 + \dots + a_nu_n + b_1w_1 + \dots + b_mw_m,\]
            and 
            \[ v = c_1u_1 + \dots + c_nu_n + d_1w_1 + \dots + d_mw_m,\]
            we have 
            \begin{align*}
                T(u+v) &= T\left((a_1+c_1)u_1 + \dots + (a_n+c_n)u_n + (b_1+d_1)w_1 + \dots + (b_m+d_m)w_m\right) \\
                       &= S\left((a_1+c_1)u_1 + \dots + (a_n+c_n)u_n \right) + (b_1+d_1)w_1 + \dots + (b_m+d_m)w_m \\
                       &= S(a_1u_1 + \dots + a_nu_n) + b_1w_1 + \dots + b_mw_m \\
                       &\quad + S(c_1u_1 + \dots + c_nu_n) + d_1w_1 + \dots + d_mw_m \\
                       &= Tu + Tv,
            \end{align*}
            Similarly, if $\lambda \in \F$, and $v = a_1u_1 + \ldots + a_nu_n + b_1w_1 + \dots + b_mw_m$, then 
            \begin{align*}
                T(\lambda v) &= T\left(\lambda (a_1u_1 + \ldots + a_nu_n + b_1w_1 + \dots + b_mw_m)\right) \\
                             &= S(\lambda a_1u_1 + dots + \lambda a_nu_n) + \lambda b_1w_1 + \dots + \lambda b_m w_m \\
                             &= S\left(\lambda (a_1u_1 + \dots + a_nu_n)\right) + \lambda (b_1w_1 + \dots + b_mw_m) \\
                             &= \lambda S(a_1u_1 + \dots + a_nu_n) + \lambda (b_1w_1 + \dots + b_mw_m) \\
                             &= \lambda \left( S(a_1u_1 + \dots + a_nu_n) + b_1w_1 + \dots + b_mw_m \right) \\
                             &= \lambda Tv,
            \end{align*}
            therefore $T$ is a linear map from $V$ to $V$.
        \end{solution}
    \item Suppose $V$ is finite-dimensional with $\dim{V} >0$, and suppose $W$ is infinite-dimensional. Prove that $\mathcal{L}(V,W)$ is infinite-dimensional.
        \begin{solution}
            Since $W$ is infinite-dimensional, from exercise \hyperlink{2A-13}{2A-13}, we know that there is a linearly independent list 
            \[ w_1, \ldots, w_m \in W\] 
            for any positive $m$. Now consider the fucntions $T_1, \ldots, T_k: V \to W$ define as:
            \[ T_i(a_1v_1 + \ldots + a_nv_n) = a_1w_1 + \dots + a_{n-1}w_{n-1} + a_nw_{n-1+i}, \quad i=1,\ldots, k\]
            where $a_1,\ldots,a_n$ are arbitrary elements in $\F$ and $v_1, \ldots, v_n$ is basis of $V$, it is easy to verify that $T_i$ is linear map 
            from $V$ to $W$, Now we will prove that $T_1, \ldots, T_k$ is linearly independent:

            Suppose that 
            \[ b_1T_1 + \dots + b_kT_k = 0, \]
            consider a special $v = v_n \in V$, we have 
                \[ b_1T_1(v_n) + \dots b_kT_k(v_n) = b_1w_{n} + \dots + b_kw_{n+k-1}, \]
            since $w_{n}, \ldots, w_{n+k-1}$ is linearly independent, we have 
            \[ b_1 = \dots = b_k = 0.\]
            
            Obeserved that we didn't the only assumption we made on $k$ is that $k$ is a postive integer, means that $\mathcal{L}(V,W)$ is infinite-dimensional.
        \end{solution}
    \item Suppose $v_1, \ldots, v_m$ is a linearly \textbf{dependent} list of vectors in $V$. Suppose also $W \neq \{0\}$. Prove that there exist 
        $w_1, \ldots, w_m \in W$ such that no $T \in \mathcal{L}(V,W)$ satisfies $Tv_k = w_k$ for each $k=1,\ldots, m$.
        \begin{solution}
            Since $W \neq \{0\}$, means there exists $w \in W$, which $w \neq 0$.
            \begin{itemize}
                \item If one of $v_1, \ldots, v_m$ is $0$, assume that $v_k=0$, consider the list $w,\ldots,w$, if such $T$ exists, we have 
                    \[ Tv_k = w ,\]
                    since $Tv_k = T0 = 0$, implies that 
                    \[w = 0,\]
                    which is contradict.
                \item If none of $v_1, \ldots, v_m$ is $0$, there is non-zero list $a_1,\ldots,a_m$ with 
                    \[ a_1v_1 + \ldots + a_mv_m = 0,\]
                    since $v_1, \ldots, v_m$ is linearly dependent, consider the list 
                    \[ w_i = 
                        \begin{cases}
                            \dfrac{1}{a_i}w \qquad& \text{if $a_i \neq 0$} \\
                            w \qquad& \text{if $a_i=0$}
                        \end{cases},
                    \]
                    if such $T$ exists, we have 
                    \[ T(a_1v_1 + \dots + a_mv_m) = kw,\] 
                    where $k$ is the number of non-zero values in $a_1,\ldots,a_m$, we have $0<k\leq m$ the left-hand side term of equation above is 
                    \[ T(a_1v_1 + \dots + a_mv_m) = T0 = 0,\]
                    implies 
                    \[ kw = 0,\]
                    means 
                    \[w=0,\]
                    which is contradict, too.
            \end{itemize}
            Thus, we can find such $w_1,\ldots, w_m$ that no $T \in \mathcal{L}(V,W)$ satisfies the condition, notice that in the prove above should provided that 
            $W$ is a vector space.
        \end{solution}
    \item Suppose $V$ is finite-dimensional with $\dim{V} \geq 2$. Prove that there exist $S, T \in \mathcal{L}(V,V)$ such that $ST \neq TS$.
        \begin{solution}
            Since $\dim{V} \geq 2$, we suppose that $u_1, u_2, \ldots u_m$ is basis of $V$, we have 
            \[ u_2, u_1+u_2, \ldots, u_m \in V,\]
            therefore there is a linear map $S$ with 
            \[ Su_1 = u_2, Su_2 = u_1 + u_2, \ldots Su_m = u_m.\]
            Similarly, we have 
            \[u_2, u_1-u_2, \ldots, u_m \in V,\]
            therefore there is a linearly map $T$ with 
            \[ Tu_1 = u_2, Tu_2 = u_1-u_2,\ldots, Tu_m = u_m,\]
            now we have 
            \[ TSu_1 = Tu_2 = u_1 - u_2, \qquad STu_1 = Su_2 = u_1 + u_2, \]
            \ie, 
            \[ TS \neq ST.\]
        \end{solution}
\end{enumerate}

\subsection{Null Spaces and Ranges}

\begin{enumerate}
    \item Give an example of a linear map $T$ such that $\dim{\Null{T}} = 3$ and $\dim{\range{T}} = 2$.
        \begin{solution}
            Consider such a linear map $T: \R^5 \to \R^2$ defined by 
            \[ T(x_1, x_2, x_3, x_4, x_5) = (x_1+x_2+x_3, x_4+x_5), \]
            we have 
            \[ \Null{T} = \{(x_1,x_2,x_3,x_4,x_5)\in \R^5: x_1+x_2+x_3=0 \text{ and } x_4 + x_5=0 \},\]
            a basis of $\Null{T}$ is 
            \[ (1,-1,0,0,0), \qquad (1,0,-1,0,0), \qquad (0,0,0,1,-1),\]
            therefore 
            \[ \dim{\Null{T}} = 3.\]
            And we can easily verify that 
            \[ \range{T} = \R^2 ,\]
            therefore 
            \[ \dim{\range{T}} = 2.\]
            Thus, the linear map $T$ we defined above satisfies the conditions.

            \textit{Remark.} We can get the example from the idear of \textbf{Fundamental Theorem of Linear Maps}:
            \[ \dim{V} = \dim{\Null{T}} + \dim{\range{T}}. \]
        \end{solution}
    \item Suppose $V$ is a vector space and $S, T \in \lmap{V}{W}$ are such that 
        \[ \range{S} \subset \null{T} .\]
        Prove that $(ST)^2 = 0$.
        \begin{solution}
            Suppose $v \in V$, since $S, T$ are linear maps from $V$ to $V$, we have 
            \[ Tv \in V ,\]
            and 
            \[ STv \in \range(S) ,\]
            since we have $\range{S} \subset \Null{T}$, we have 
            \[ STv \in \Null{T} ,\]
            \ie,
            \[ TSTv = 0, \]
            with $S$ is a linear map means $S0 = 0$, therefore 
            \[ STSTv = 0, \]
            for any $v \in V$, thus 
            \[ (ST)^2 = 0. \]
        \end{solution}
    \item Suppose $v_1, \ldots, v_m$ is a list of vectors in $V$. Define $T \in \lmap{\F^m}{V}$ by 
        \[ T(z_1, \ldots, z_m) = z_1v_1 + \dots + z_mv_m .\]
        \begin{enumerate}[label=(\alph*)]
            \item What property of $T$ correspnods to $v_1, \ldots, v_m$ spanning $V$?
            \item what property of $T$ corresponds to $v_1, \ldots, v_m$ being linearly independent?
        \end{enumerate}
        \begin{solution}
            \begin{enumerate}[label=(\alph*)]
                \item Since $v_1, \ldots, v_m$ spans $V$, means 
                    \[ \range{T} = V .\]
                \item If $v_1, \ldots, v_m$ is linearly independent, means 
                    \[ \Null{T} = \{0\} .\]
            \end{enumerate}
        \end{solution}
    \item Show taht 
        \[ \{T \in \lmap{\R^5}{\R^4}: \dim{\Null{T}} > 2 \} \]
        is not subspace of $\lmap{\R^5}{\R^4}$.
        \begin{solution}
            Consider the linear maps $T_1, T_2 \in \lmap{\R^5}{\R^4}$ defined by 
            \[ T_1(x_1, x_2, x_3, x_4, x_5) = (0,0,x_3,x_4),\qquad T_2(x_1,x_2,x_3,x_4,x_5) = (0,0,-x_3,x_4) ,\]
            they are both linear maps, and 
            \[ \dim{\Null{T_1}} = \dim{\Null{T_2}} = 3 > 2,\]
            but we have 
            \[ (T_1 + T_2) (x_1, x_2, x_3, x_4, x_5) = (0,0,0,x_4) \]
            means 
            \[ \dim{\Null{(T_1 + T_2)}} = 2 \]
            which is not in this set, \ie, it is not a subspace.
        \end{solution}
    \item Give an example of a linear map $T: \R^4 \to \R^4$ such that 
        \[ \range{T} = \Null{T} .\]
        \begin{solution}
            Consider the function $T: \R^4 \to \R^4$ defined by 
            \[ T(x_1,x_2, x_3, x_4) = (0,0,x_1,x_2) .\]
            We will show that $T$ satisfies the conditions: 

            It is easy to verify that $T$ is a linear map. And we have 
            \[ \range{T} = \{(0,0,x,y) \in \R^4: x,y \in \R \}, \]
            also we have 
            \[ \Null{T} = \{(0,0,x,y) \in \R^4: x,y \in \R \} ,\]
            \ie, 
            \[ \range{T} = \Null{T}. \]
        \end{solution}
    \item Prove that there does not exist a linear map $T: \R^5 \to \R^5$ such that 
        \[ \range{T} = \Null{T}. \]
        \begin{solution}
            Assume that $T$ is such a linear map, according to Fundamental Theorem we have 
            \[ 5 = \dim{\range{T}} + \dim{\Null{T}},\]
            and since $\range{T} = \Null{T}$, implies 
            \[ \dim{\range{T}} = \dim{\Null{T}}, \]
            \ie, 
            \[ \dim{\range{T}} = \dim{\Null{T}} = 2.5, \]
            which is impossible since the dimension of vector space should be integer.
        \end{solution}
    \item Suppose $V$ and $W$ are finite-dimensional with $2 \leq \dim{V} \leq \dim{W}$. Show that 
        $\{ T \in \lmap{V}{W} : T \text{ is not injective}\}$ is not a subspace of $\lmap{V}{W}$.
        \begin{solution}
            Suppose $\dim{V} = n$, and $v_1, \ldots, v_m$ is a basis of $V$, since $\dim{W} \geq \dim{V}$, there exists
            linearly independence list $w_1, \ldots, w_n \in W$, we define the function $T_1: V \to W$ by 
            \[ T_1v_1=w_1,\ \ldots,\ T_1v_{n-1}=w_{n-1},\ T_1v_n = 0, \]
            and $T_2 : V \to W$ by 
            \[ T_2v_1 = w_1,\ \ldots,\ T_2v_{n-1} = 0,\ T_2v_n = w_n .\]
            For $T_1$ we have 
            \[ T_1(0) = 0, \qquad T_1(v_n) = 0, \]
            means $T_1$ is not injective, with same method we have $T_2$ is not injective, now we consider $T_1 + T_2$, for 
            any element in $V$, denoted by $v = a_1v_1 + \dots + v_n$, we have 
            \begin{align*}
                (T_1+T_2)(v) &= T_1v + T_2v \\
                             &= T_1(a_1v_1 + \dots + a_nv_n) + T_2(a_1v_1+\dots+a_nv_n) \\
                             &= 2a_1w_1 + \ldots + 2a_{n-2}w_{n-2} + a_{n-1}w_{n-1} + a_nw_n.
            \end{align*}
            Since $w_1, \ldots, w_n$ is linearly independent, means the solution of 
            \[ (T_1+T_2)(v) = 0\]
            has only one solution 
            \[ v = 0,\]
            \ie, linear map $T_1 + T_2$ is injective, which means the set is not closed under addition, thus it is not subspace.
        \end{solution}
    \item Suppose $V$ and $W$ are finite-dimensional with $\dim{V} \geq \dim{W} \geq 2$. Show that 
        $\{ T \in \lmap{V}{W}: T \text{ is not surjective} \}$ is not a subspace of $\lmap{V}{W}$.
        \begin{solution}
            This question is quite similar with the one above, suppose that $\dim{V}=n$ and $v_1, \ldots, v_n$ is a basis of 
            $V$, and $\dim{W} = m$, $w_1, \ldots, w_m$ is a basis of $W$. 
            $T_1, T_2 \in \lmap{V}{W}$ which defined by
            \[ T_1(a_1v_1 + \dots + a_nv_n) = a_1w_1 + a_3w_3 + \dots + a_mw_m,\]
            and 
            \[ T_2(a_1v_1 + \dots + a_nv_n) = a_2w_2 + a_3w_3 + \dots + a_mw_m,\]
            where $a_1,\ldots, a_n \in \F$ are arbitrary.

            It is easy to verify that 
            \[ w_2 \notin \range{T_1}, \qquad w_1 \notin \range{T_2}, \]
            \ie, $T_1, T_2$ are not surjective.

            But for $T_1+T_2$, we have 
            \[ (T_1+T_2)(a_1v_1 + \dots + a_nv_n) = a_1w_1 + a_2w_2 + 2a_3w_3 + \dots + 2a_mw_m ,\]
            since $w_1, \ldots, w_m$ is a basis of $W$, implies that $T_1 + T_2$ is surjective, which means the vector space defined 
            in question is not closed under addition, \ie, it is not a subspace.
        \end{solution}
    \item \hypertarget{3B-9}{Suppose} $T \in \lmap{V}{W}$ is injective and $v_1, \ldots, v_n$ is linearly independent in $V$.
        Prove that $Tv_1, \ldots, Tv_n$ is linearly independent in $W$.
        \begin{solution}
            Assume that $Tv_1, \ldots, Tv_n$ is linearly dependent, implies there are nonzero $a_1, \ldots, a_n \in \F$ with 
            \[ a_1Tv_1 + \dots + a_nTv_n = 0,\]
            as we have 
            \begin{align*}
                a_1Tv_1 + \dots + a_nTv_n &= T(a_1v_1) + \dots + T(a_nv_n) \\
                                          &= T(a_1v_1 + \dots + a_nv_n) ,
            \end{align*}
            \ie, 
            \[ T(a_1v_1 + \dots + a_nv_n) = 0, \]
            since $T$ is injective, means 
            \[ a_1v_1 + \dots + a_nv_n = 0,\]
            with $v_1, \ldots, v_n$ is linearly independent, implies
            \[ a_1 = \dots = a_n = 0, \]
            which is contradict with our assumption.
        \end{solution}
    \item Suppose $v_1, \ldots, v_n$ spans $V$ and $T \in \lmap{V}{W}$. Prove that the list $Tv_1, \ldots, Tv_n$ spans $\range{T}$.
        \begin{solution}
            This question is equivalent to prove that 
            \[ \mspan{Tv_1, \ldots, Tv_n} = \range{T} .\]
            First we show that 
            \[ \mspan{Tv_1, \ldots, Tv_n} \subseteq \range{T} ,\]
            suppose $w \in \mspan{Tv_1, \ldots, Tv_n}$, implies $w$ can be written in 
            \[ w = a_1Tv_1 + \dots + a_nTv_n,\]
            here $a_1, \ldots, a_n \in \F$, which means 
            \begin{align*}
                w &= a_1Tv_1 + \dots + a_nTv_n \\
                  &= T(a_1v_1) + \dots + T(a_nv_n) \\
                  &= T(a_1v_1 + \dots + a_nv_n),
            \end{align*}
            since $v_1, \ldots, v_n$ spans $V$, means there exists $v \in V$ with 
            \[ v = a_1v_1 + \dots + a_nv_n,\]
            \ie, we have 
            \[ w = T(v) ,\]
            means 
            \[ w \in \range{T} .\]

            Second we show that 
            \[ \range{T} \subseteq \mspan{Tv_1, \ldots, Tv_n} ,\]
            suppose $w \in \range{T}$ with $w = T(v), v \in V$, since $v_1, \ldots, v_n$ spans $V$, we can form $v$ as 
            \[ v = a_1v_1 + \dots + a_nv_n, \]
            where $a_1, \ldots, a_n \in \F$, therefore 
            \begin{align*}
                w &= Tv \\
                  &= T(a_1v_1 + \dots + a_nv_n) \\
                  &= T(a_1v_1) + \dots + T(a_nv_n) \\
                  &= a_1Tv_1 + \dots + a_nTv_n,
            \end{align*}
            \ie,
            \[ w \in \mspan{Tv_1, \ldots, Tv_n}.\]

            Thus we conclude that 
            \[ \mspan{Tv_1, \ldots, Tv_n} = \range{T} .\]
        \end{solution}
    \item Suppose $S_1, \ldots, S_n$ are injective linear maps such that $S_1S_2\dots S_n$ makes sense. Prove that $S_1S_2\dots S_n$ is injective.
        \begin{solution}
            Assume that $S_1S_2\dots S_n$ is not injective, means there exists $u \neq 0$ with 
            \[ S_1S_2\dots S_nu = 0, \]
            since 
            \[ S_1S_2\dots S_nu = S_1(S_2\dots S_nu) \]
            and $S_1$ is injective, implies 
            \[ S_2\dots S_nu = 0,\]
            continue this iteration, we conclude that 
            \[ S_nu = 0 \]
            which is contradict since $S_n$ is injective.
        \end{solution}
    \item Suppose that $V$ is finite-dimensional and that $T \in \lmap{V}{W}$. Prove that there exists a subspace $U$ of $V$ such that 
        $U \cap \Null{T} = \{0\}$ and $\range{T} = \{Tu: u \in U\}$.
        \begin{solution}
            Since $V$ is finite-dimensional, suppose $\dim{\Null{T}} = m$, and
            $v_1,\ldots,v_m$ is a basis of $\Null{T}$, since $v_1, \ldots, v_m$ is linearly independent list of 
            $V$, it can be extended to be a basis of $V$, suppose that 
            \[ v_1, \ldots, v_m, u_1, \ldots, u_n\] 
            to be the extended basis of $V$, \ie, $\dim{V} = n+m$.

            Now consider subspace $U$ obtained by 
            \[ U = \mspan{u_1, \ldots, u_n}. \]

            It is easy to verify that $U \cap \Null{T} = \{0\}$ since $v_1, \ldots, v_m, u_1, \ldots, u_n$ is linearly independent.
            Now we will show that $Tu_1, \ldots, Tu_n$ is linearly independent, or we have nonzero $a_1, \ldots, a_n \in \F$ with 
            \[ a_1Tu_1 + \dots + a_nTu_n = 0,\]
            implies that 
            \[ T(a_1u_1 + \dots + a_nu_n) = 0, \]
            means 
            \[ a_1u_1 + \dots + a_nu_n \in \Null{T}, \]
            and with $v_1, \ldots, v_m$ is basis of $\Null{T}$, we have 
            \[ a_1u_1 + \dots + a_nu_n = b_1v_1 + \dots + b_mv_m, \]
            since $v_1, \ldots, v_m, u_1, \ldots, u_n$ is linearly independent, we have 
            \[ a_1 = \dots = a_n = 0,\]
            which is contradict.

            From above we have $Tu_1, \ldots, Tu_n \in \range{T}$ is linearly independent, and with Fundamental Theorem we have 
            \[ \dim{\range{T}} = \dim{V} - \dim{\Null{T}} = n, \]
            which means $Tu_1, \ldots, Tu_n$ is a basis of $\range{T}$, \ie, 
            \[ \range{T} = \mspan{Tu_1, \ldots, Tu_n}, \]
            since $\mspan{Tu_1, \ldots, Tu_n} = \{Tu: u \in U\}$ we have 
            \[ \range{T} = \{Tu: u \in U\} .\]
        \end{solution}
    \item Suppose $T$ is a linear map from $\F^4$ to $\F^2$ such that
        \[ \Null{T} = \{(x_1,x_2,x_3,x_4) \in \F^4: x_1=5x_2 \text{ and } x_3 = 7x_4 \}. \]
        Prove that $T$ is surjective.
        \begin{solution}
            We know that 
            \[ (5,1,0,0), \qquad (0,0,7,1)\]
            is a basis of $\Null{T}$, means 
            \[ \dim{\Null{T}} = 2, \]
            with $\dim{\F^4} = 4$ and Fundamental Theorem we have 
            \[ \dim{\range{T}} = 4 - 2 = 2,\]
            since $\dim{\F^2}=2$, and $\range{T} \subseteq \F^2$, we have 
            \[ \range{T} = \F^2, \]
            \ie, it is surjective.
        \end{solution}
    \item Suppose $U$ is a $3$-dimensional subspace of $\R^8$ and that $T$ is a linear map from $\R^8$ to $\R^5$
        such that $\Null{T} = U$. Prove that $T$ is surjective.
        \begin{solution}
            This is similar to the question above, we can use the same method to prove it, with Fundamental Theorem we have 
            \[ \dim{\range{T}} = \dim{\R^8} - \dim{\Null{T}} = 8 - 3 = 5, \]
            and since 
            \[ \range{T} \subseteq \R^5, \qquad \dim{\range{T}} = \dim{\R^5} \]
            we have 
            \[ \range{T} = \R^5 .\]
            Or there is an element with $u \in \R^5$ and $u \notin \range{T}$, mean $u$ and the basis of $\range{T}$ is linearly independent,
            implies that the dimension of $\R^5$ is greater than 5, which is contradict.
        \end{solution}
    \item Prove that there does not exist a linear map from $\F^5$ to $\F^2$ whose null space equals
        \[ \{ (x_1, x_2, x_3, x_4, x_5) \in \F^5: x_1 = 3x_2 \text{ and } x_3 = x_4 = x_5 \}.\]
        \begin{solution}
            We can figure out that 
            \[ (3,1,0,0,0), \qquad (0,0,1,1,1) \]
            is a basis of the vector space above.

            Assume that such linear map $T$ exists, we have 
            \[ \dim{\Null{T}} = 2, \]
            with Fundamental Theorem we have 
            \[ \dim{\range{T}} = \dim{\F^5} - \dim{\Null{T}} = 5 - 2 = 3 ,\]
            since $\range{T} \subseteq \F^2$, implies 
            \[ \dim{\range{T}} \leq \dim{\F^2} = 2, \]
            thus, there is no such linear map.
        \end{solution}
    \item Suppose there exist a linear map on $V$ whose null space and range are both finite-dimensional. Prove 
        that $V$ is finite-dimensional.
        \begin{solution}
            With Fundamental Theorem we have 
            \[ \dim{V} = \dim{\Null{T}} + \dim{\range{T}} ,\]
            \ie, $V$ is finite-dimensional.
        \end{solution}
    \item Suppose $V$ and $W$ are both finite-dimensional. Prove that there exists an injective linear map from 
        $V$ to $W$ if and only if $\dim{V} \leq \dim{W}$.
        \begin{solution}
            First we show that if $\dim{V} \leq \dim{W}$, then such linear map exists:
            Since both $V$ and $W$ are finite-dimensional, suppose that $v_1, \ldots, v_n$ is a basis of $V$, 
            and with $\dim{V} \leq \dim{W}$, we know there exist a linearly independent list $w_1,\ldots,w_n \in W$,
            consider the function $T \in \lmap{V}{W}$ defined by 
            \[ T(a_1v_1 + \dots + a_nv_n) = a_1w_1 + \dots + a_nw_n ,\]
            where $a_1, \ldots, a_n \in \F$ are arbitrary.

            Since $w_1, \ldots, w_n$ is linearly independent, means 
            \[ a_1w_1 + \dots + a_nw_n = 0\]
            holds only when $a_1=\dots=a_n=0$, \ie,
            \[ \Null{T} = \{0\},\]
            means $T$ is injective.

            The reverse is obviously since from the conclusion of the textbook, we know that when $\dim{V} > \dim{W}$, there 
            is no injective linear map from $V$ to $W$.
        \end{solution}
    \item Suppose $V$ and $W$ are both finite-dimensional. Prove that there exists a surjective linear map from 
        $V$ onto $W$ if and only if $\dim{V} \geq \dim{W}$.
        \begin{solution}
            This is quite similar to the previous one, suppose that $w_1, \ldots, w_m$ is the basis of $W$, and 
            $v_1, \ldots, v_n$ is the basis of $V$, since we have $m \leq n$, consider the linear map $T \in \lmap{V}{W}$ defined by 
            \[ T(a_1v_1 + \dots + a_nv_n) = a_1w_1 + \dots + a_mw_m ,\]
            where $a_1, \ldots, a_n \in \F$ are arbitrary.

            Since 
            \[ \mspan{w_1, \ldots, w_m} = W,\]
            we have 
            \[ \range{T} = W,\]
            \ie, $T$ is surjective.
        \end{solution}
    \item Suppose $V$ and $W$ are finite-dimensional and that $U$ is subspace of $V$. Prove that there exists $T \in \lmap{V}{W}$ such that 
        $\Null{T} = U$ if and only if $\dim{U} \geq \dim{V} - \dim{W}$.
        \begin{solution}
            First we show that if $\dim{U} \geq \dim{V} - \dim{W}$, then such linearly map $T$ exist, suppose that $u_1, \ldots, u_m$ is a basis of $U$,
            since $U \subseteq V$, means that $u_1,\ldots, u_m$ can be extended to be a basis of $V$, suppose the extended basis of $V$ is 
            \[ u_1, \ldots, u_m, v_1, \ldots, v_n ,\] 
            therefore we have 
            \[ \dim{U} = m, \qquad \dim{V} = m+n,\]
            means 
            \[ \dim{W} \geq \dim{V} - \dim{U} = m+n-m = n.\]
            Therefore there are linearly independent list $w_1, \ldots, w_n \in W$, consider the linear map $T \in \lmap{V}{W}$ defined by
            \[ T(a_1u_1+\dots+a_mu_m+b_1v_1+\dots+b_nv_n) = b_1w_1 + \dots + b_nw_n, \]
            where $a_1,\ldots,a_m,b_1,\ldots,b_n \in \F$ are arbitrary.

            Let 
            \[ b_1w_1 + \dots + b_nw_n = 0, \]
            implies that 
            \[ b_1 = \dots =b_n = 0,\]
            means 
            \[ \Null{T} = \mspan{u_1, \ldots, u_m} = U. \]

            Seconde we show the reverse: if $\dim{U} < \dim{V} - \dim{W}$, then there is no such linear map:
            according to Fundamental Theorem we have 
            \begin{align*}
                \dim{\Null{T}} &= \dim{V} - \dim{\range{T}}\\
                               &> \dim{V} - \dim{W}\\
                               &> \dim{U},
            \end{align*}
            means there is no $T \in \lmap{V}{W}$ that $\Null{T} = U$.
        \end{solution}
    \item Suppose $W$ is finite-dimensional and $T \in \lmap{V}{W}$. Prove that $T$ is injective if and only if there exists $S \in \lmap{W}{V}$ 
        such that $ST$ is the identity map on $V$.
        \begin{solution}
            First, we will show that if $T$ is injective, then such $S$ exists: since $T$ is injective, then $\dim{V} \leq \dim{W}$, suppose that 
            $v_1, \ldots, v_n$ is a basis of $V$, according to exercise \hyperlink{3B-9}{3B-9}, we know that 
            \[ Tv_1, \ldots, Tv_n \]
            is linearly independent in $W$, which means it can extend to be a basis of $W$, suppose the extended basis is 
            \[ Tv_1, \ldots, Tv_n, w_1, \ldots, w_m \]
            consider the function $S \in \lmap{W}{V}$ defined by 
            \[ S(a_1Tv_1 + \dots + a_nTv_n + b_1w_1 + \dots + b_mw_m) = a_1v_1 + \dots + a_nv_n ,\]
            where $a_1,\ldots, a_n,b_1,\ldots,b_m \in \F$ are arbitrary. For any $v \in V$, we can form $v$ as 
            \[ v = a_1v_1 + \dots + a_nv_n, \] 
            therefore 
            \begin{align*}
                STv &= ST(a_1v_1 + \dots + a_nv_n) \\
                    &= S(Ta_1v_1 + \dots + Ta_nv_n) \\
                    &= S(a_1Tv_1 + \dots + a_nTv_n) \\
                    &= a_1v_1 + \dots + a_nv_n \\
                    &= v,
            \end{align*}
            \ie, $ST$ is identity map on $V$.

            Second, we will show the reverse direction: If $ST$ is identity then $T$ is injective, or, assume that $u \neq 0$ with 
            \[ Tu = 0, \]
            we have 
            \[ STu = S0 = 0,\]
            but since $ST$ is identity map, we have 
            \[ STu = u,\]
            implies that
            \[ u = 0, \]
            which is contradict to our assumption.
        \end{solution}
    \item Suppose $W$ is finite-dimensional and $T \in \lmap{V}{W}$. Prove that $T$ is surjective if and only if there exists $S \in \lmap{W}{V}$ 
        such that $TS$ is the identity map on $W$.
        \begin{solution}
            First, we will show that if $T$ is surjective, then such $S$ exists: Suppose $w_1, \ldots, w_n$ is a basis of $W$, since $T$ is surjective, 
            means we can find $v_1, \ldots, v_n \in V$ that 
            \[ Tv_i = w_i, \qquad i=1, \ldots, n,\]
            Consider the function $S \in \lmap{W, V}$ defined by 
            \[ S(a_1w_1 + \dots + a_nw_n) = a_1v_1 + \dots + a_nv_n ,\]
            where $a_1, \ldots, a_n \in \F$ are arbitrary. Now for any $w \in W$, we can form $w$ as 
            \[ w = a_1w_1 + \dots + a_nw_n,\]
            then 
            \begin{align*}
                TSw &= TS(a_1w_1 + \dots + a_nw_n) \\
                    &= T(a_1v_1 + \dots + a_nv_n) \\
                    &= Ta_1v_1 + \dots + Ta_nv_n \\
                    &= a_1Tv_1 + \dots + a_nTv_n \\
                    &= a_1w_1 + \dots + a_nw_n \\
                    &= w,
            \end{align*}
            \ie, $TS$ is identity map on $W$.

            Second, we will show the reverse direction: Assume that $T$ is not surjective, means that there is $w \in W$, that $w \notin \range{T}$,
            but we have 
            \[ T(Sw) = w, \qquad Sw \in V,\]
            means 
            \[ w \in \range{T}, \]
            which is contradict.
        \end{solution}
    \item Suppose $U$ and $V$ are finite-dimensional vector spaces and $S \in \lmap{V}{W}$ and $T \in \lmap{U}{V}$. Prove that 
        \[ \dim{\Null{ST}} \leq \dim{\Null{S}} + \dim{\Null{T}} .\]
        \begin{solution}
            Suppose that $u_1, \ldots, u_n$ is a basis of $\Null{T}$, it is easy to verify that $u_i \in \Null{ST}$ for $i=1,\dots,n$ 
            since 
            \[ STu_i = S0 = 0, \qquad i=1,\ldots,n ,\]
            therefore it can extended to be a basis of $\Null{ST}$, suppose to be 
            \[ u_1, \ldots, u_n, v_1, \ldots, v_m .\]
            Now we will show that $Tv_i \in \Null{S}$ for $i=1,\ldots,m$ and $Tv_1, \ldots, Tv_n$ is linearly independent:
            Obviously that $Tv_i \in \Null{S}$ since 
            \[ STv_i = S(Tv_i) = 0,\]
            assume the list is not linearly independent, means there exists nonzero $a_1, \ldots, a_m \in \F$ with 
            \[ a_1Tv_1 + \dots + a_mTv_m = 0, \]
            since 
            \[ a_1Tv_1 + \dots + a_mTv_m = T(a_1v_1 + \dots + a_mv_m ),\]
            means 
            \[ a_1v_1 + \dots + a_mv_m \in \Null{T} ,\]
            and with $u_1, \ldots, u_n$ is a basis of $\Null{T}$, means there exists $b_1, \ldots, b_n \in \F$ that 
            \[ a_1v_1 + \dots + a_mv_m = b_1u_1 + \dots + b_nu_n ,\]
            and with $u_1, \ldots, u_n, v_1, \ldots, v_m$ is linearly independent, we have 
            \[ a_1 = \dots = a_m = b_1 = \dots = b_n = 0 ,\]
            which is a contradict, therefore $Tv_1, \ldots, Tv_m$ is linearly independent in $\Null{S}$, 
            \[ m \leq \dim{\Null{S}}, \]
            with $\dim{\Null{ST}} = n + m$ and $\dim{\Null{T}} = n$ we have 
            \[ \dim{\Null{ST}} \leq \dim{\Null{T}} + \dim{\Null{S}}.\]
        \end{solution}
    \item Suppose $U$ and $V$ are finite-dimensional vector space and $S \in \lmap{V}{W}$ and $T \in \lmap{U}{V}$. Prove that 
        \[ \dim{\range{ST}} \leq \min{\{ \dim{\range{S}}}, \dim{\range{T}} \}.\]
        \begin{solution}
            First we will show that 
            \[ \dim{\range{ST}} \leq \dim{\range{S}}, \]
            this is easy, since the basis of $\range{ST}$ must be linearly independent in $\range{S}$, therefore the inequality holds.

            Second we will show that 
            \[ \dim{\range{ST}} \leq \dim{\range{T}} ,\]
            suppose that $u_1, \ldots, u_n$ is basis of $\range{T}$, we will show that $Su_1, \ldots, Su_n$ spans $\range{S}$, for 
            any $s \in \range{ST}$, suppose that 
            \[ s = STu, u \in U ,\]
            since $Tu \in \range{T}$, we can form it as 
            \[ Tu = a_1u_1 + \dots + a_nu_n, \]
            here $a_1, \ldots, a_n \in \F$, therefore above equation can be written as 
            \begin{align*}
                s &= STu \\
                  &= S(a_1u_1 + \dots + a_nu_n) \\
                  &= a_1Su_1 + \dots + a_nSu_n,
            \end{align*}
            which means any $s \in \range{ST}$ can be formed as a linear combination of $Su_1, \ldots, Su_n$, \ie, $Su_1, \ldots, Su_n$ spans 
            $\range{ST}$, therefore it can be reduced to be a basis of $\range{TS}$, means 
            \[ \dim{\range{ST}} \leq n = \range{T} .\]
        \end{solution}
    \item Suppose $W$ is finite-dimensional and $T_1, T_2 \in \lmap{V}{W}$. Prove that $\Null{T_1} \subset \Null{T_2}$ if and only if there exists 
        $S \in \lmap{W}{W}$ such that $T_2 = ST_1$.
        \begin{solution}
            First we will show that if $\Null{T_1} \subset \Null{T_2}$, then such linear map $S$ exists:
            suppose $u_1, \ldots, u_n$ is basis of $\Null{T_1}$, since $\Null{T_1} \subset \Null{T_2}$, means $u_1, \ldots, u_n$ is 
            linearly independent in $\Null{T_2}$, it can extended to be a basis of $\Null{T_2}$, suppose to be 
            \[ u_1, \ldots, u_n, v_1, \ldots, v_m, \]
            this list can be extended to be a basis of $U$, suppose to be 
            \[ u_1, \ldots, u_n, v_1, \ldots, v_m, s_1,\ldots, s_p \]
            we will show that 
            \[ T_1v_1, \ldots, T_1v_m, T_1s_1, \ldots, T_1s_p  \]
            is linearly independent in $W$, obviously that $T_1v_i \in W$ for $i = 1, \ldots, m$ and $T_1s_i \in W$ for $i = 1,\ldots, p$,
            assume that it is not linearly independent, 
            we have nonzero $a_1, \ldots, a_m, b_1, \ldots, b_p \in \F$ with 
            \[ a_1T_1v_1 + \dots + a_mT_1v_m  + b_1T_1s_1 + \dots + b_pT_1s_p = 0, \]
            since 
            \[ a_1T_1v_1 + \dots + a_mT_1v_m + b_1T_1s_1 + \dots + b_pT_1s_p = T_1(a_1v_1 + \dots + a_mv_m + b_1s_1 + \dots + b_ps_p) = 0, \]
            means 
            \[ a_1v_1 + \dots + a_mv_m + b_1s_1 + \dots + b_ps_p \in \Null{T_1}, \]
            with $u_1, \ldots, u_n$ is a basis of $\Null{T_1}$, we have 
            \[ a_1T_1v_1 + \dots + a_mT_1v_m + b_1s_1 + \dots + b_ps_p = c_1u_1 + \dots + c_nu_n ,\]
            with $u_1, \ldots, u_n, v_1, \ldots, v_m, s_1, \ldots, s_p$ is linearly independent, we have
            \[ a_1 = \dots = a_m = b_1 = \dots = b_p = 0,\]
            which is contradict, therefore $T_1v_1, \ldots, T_1v_m, T_1s_1, \ldots, T_1s_p$ is linearly independent in $W$, 
            implies it can extended to be a basis of $W$, 
            suppose the basis to be 
            \[ T_1v_1, \ldots, T_1v_m, T_1s_1, \ldots, T_1s_p, w_1, \ldots, w_k ,\]
            consider the function $S: W \to W$ defined by 
            \begin{align*}
                &S(a_1T_1v_1 + \dots + a_mT_1v_m + b_1T_1s_1 + \dots + b_pT_1s_p + c_1w_1 + \dots + c_kw_k) \\
                &\qquad = b_1T_2s_1 + \dots + b_pT_2s_p + c_1w_1 + \dots + c_kw_k
            \end{align*}
            it is easy to verify that $S \in \lmap{W}{W}$, now we will show that 
            \[ T_2 = ST_1 ,\]
            for $u \in U$. we can form $u$ as 
            \[ u = a_1u_1 + \dots + a_nu_n + b_1v_1 + \dots + b_mv_m + \dots + c_1s_1 + \dots + c_ps_p ,\]
            therefore 
            \begin{align*}
                T_2u &= T_2(a_1u_1 + \dots + a_nu_n + b_1v_1 + \dots + b_mv_m + \dots + c_1s_1 + \dots + c_ps_p) \\
                     &= T_2(c_1s_1 + \dots + c_ps_p),
            \end{align*}
            and 
            \begin{align*}
                ST_1u &= ST_1(a_1u_1 + \dots + a_nu_n + b_1v_1 + \dots + b_mv_m + \dots + c_1s_1 + \dots + c_ps_p) \\
                        &= S(b_1T_1v_1 + \dots + b_mT_1v_m + c_1T_1s_1 + \dots + c_pT_1s_p) \\
                        &= c_1T_2s_1 + \dots + c_pT_2s_p \\
                        &= T_2(c_1s_1 + \dots + c_ps_p),
            \end{align*}
            which means 
            \[ T_2 = ST_1 .\]

            Now we will show it in the reverse direction: If $T_2 = ST_1$, then for any $u \in \Null{T_1}$, we have 
            \[ T_2u = ST_1u = S0 = 0,\]
            means 
            \[ u \in \Null{T_2}, \]
            \ie, $\Null{T_1} \subset \Null{T_2}$.
        \end{solution}
    \item Suppose $V$ is finite-dimensional and $T_1,T_2 \in \lmap{V}{W}$. Prove that $\range{T_1} \subset \range{T_2}$ if and only if 
        there exists $S \in \lmap{V}{V}$ such that $T_1 = T_2S$.
    \item Suppose $D \in \lmap{\poly{\R}}{\poly{\R}}$ is such that $\deg{Dp} = (\deg{p}) - 1$ for every nonconstant polynomial $p \in \poly{\R}$.
        Prove that $D$ is surjective.
    \item Suppose $p \in \poly{\R}$. Prove that there exists a polynomial $q \in \poly{\R}$ such that $5q''+3q'=p$.
    \item Suppose $T \in \lmap{V}{W}$, and $w_1, \ldots, w_m$ is a basis of $\range{T}$. Prove that there exist $\varphi_1, \ldots, \varphi_m \in \lmap{V}{\F}$
        such that 
        \[ Tv = \varphi_1(v)w_1 + \dots + \varphi_m(v)w_m \]
        for every $v \in V$.
    \item Suppose $\varphi \in \lmap{V}{\F}$. Suppose $u \in V$ is not in $\Null{\varphi}$. Prove that 
        \[ V = \Null{\varphi} \oplus \{au: a\in \F \}. \]
    \item Suppose $\varphi_1$ and $\varphi_2$ are linear map from $V$ to $\F$ that have the same null space. 
        Show that there exists a constant $c \in \F$ such that $\varphi_1 = c\varphi_2$.
    \item Give an example of two linear maps $T_1$ and $T_2$ from $\R^5$ to $\R^2$ that have the same null space but are 
        such that $T_1$ is not a scalar multiple of $T_2$. 
\end{enumerate}

\end{document}
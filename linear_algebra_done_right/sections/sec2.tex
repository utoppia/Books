\section{Finit-Dimensional Vector Spaces}

\subsection{Span and Linear Independence}

\begin{enumerate}
    \item Suppose $v_1, v_2, v_3, v_4$ spans $V$. Prove that the list 
        \[ v_1-v_2, v_2-v_3, v_3-v_4, v_4\]
        also spans $V$.
        \begin{solution}
            If $u \in \mspan{v_1, v_2, v_3,v_4}$, \ie 
            \begin{align*}
                u &= a_1v_1 + a_2v_2 + a_3v_3 + a_4v_4 \\
                  &= a_1(v_1-v_2) + (a_2+a_1)(v_2-v_3)+(a_3+a_2+a_1)(v_3-v_4) + (a_4+a_3+a_2+a_1)v_4,
            \end{align*}
            therefore $u \in \mspan{ v_1-v_2, v_2-v_3, v_3-v_4, v_4}$.

            Now suppose $u \in \mspan{ v_1-v_2, v_2-v_3, v_3-v_4, v_4}$, \ie 
            \begin{align*}
                u &= a_1(v_1-v_2) + a_2(v_2-v_3) + a_3(v_3-v_4) + a_4v_4 \\
                  &= a_1v_1 + (a_2-a_1)v_2 + (a_3-a_2)v_3 + (a_4-a_3)v_4
            \end{align*}
            therefore $u \in \mspan{v_1, v_2, v_3, v_4}$.

            Thus we have 
            \[ \mspan{v_1, v_2, v_3, v_4} = \mspan{ v_1-v_2, v_2-v_3, v_3-v_4, v_4}.\]
        \end{solution}
    \item Verify the assertions in Example 2.18.
        \begin{solution}
            \begin{enumerate}[label=(\alph*)]
                \item \textcolor{black}{A list $v$ of one vector $v \in V$ is linearly independent if and only if $v \neq 0$.}
                    First suppose $v$ is linearly independent and $v = 0$, for any $a \in \F$, we have $av=0$, therefore $v$ is linearly dependent, which is
                    a contradict.

                    If $v \neq 0$, the solution of equation $av = 0$, for $a \in \F$ is $a = 0$, therefore $v$ is linearly independent.
                \item \textcolor{black}{A list of two vectors in $V$ is linearly independent if and only if neither vector is a scalar multiple of the other.}
                    Suppose these two vectors are $u$ and $v$. If one is a scalar multiple of another, assume that $u = \lambda v$ for $\lambda \neq 0$,
                    we have 
                    \[ -u + \lambda v = 0\]
                    which means that $\{u,v\}$ is not linearly independent.

                    Now we assume that neither of $u$ and $v$ is a scalar multiple of the other and $\{u,v\}$ is linearly dependent, which menns there are nonzero 
                    $a_1, a_2$ with 
                    \[ a_1 u + a_2v = 0,\]
                    we can easy suppose that $a_1 \neq 0$, we have 
                    \[ u = -(a_2/a_1)v\]
                    which is a contradict to our assumption.
                \item \textcolor{black}{$(1,0,0,0),(0,1,0,0),(0,0,1,0)$ is linearly independent in $\F^4$.}
                    Let $a_1, a_2, a_3$ to be the coeffecients which make the sum of $(1,0,0,0),(0,1,0,0),(0,0,1,0)$ be zero, \ie, 
                    \[ a_1(1,0,0,0) + a_2(0,1,0,0)+a_3(0,0,1,0) =0,\]
                    which is 
                    \[ (a_1,a_2,a_3,0) = 0, \]
                    therefore 
                    \[ a_1 =0, \qquad a_2=0, \qquad a_3=0.\]
                \item \textcolor{black}{The list $1, z, \ldots, z^m$ is linearly independent in $\mathcal{P}(\F)$ for each nonnegative integer $m$.}
                    When $m=0$, this list is $1$ which is linearly independent, now suppose the list is linearly independent for $m=0,\ldots k$,
                    let 
                    \[ a_0 + a_1z + \dots + a_{k+1}z^{k+1} = 0,\]
                    let $z = 0$ we have $a_0=0$, the equation above is same with 
                    \[ a_1 z + \dots + a_{k+1}z^{k+1} = z(a_1 + \dots + a_{k+1}z^k) = 0,\]
                    which is same that 
                    \[ a_1 + \dots + a_{k+1}z^k = 0,\]
                    since $1, \ldots, z^{k}$ is linearly independent as we assumed, we have 
                    \[ a_1 = \dots = a_{k+1} = 0, \]
                    with 
                    \[ a_0 = 0, \]
                    we conclude that $1 ,\dots, z^{k+1}$ is linearly independent.
            \end{enumerate}
        \end{solution}
    \item Find a number $t$ such that 
        \[ (3,1,4), (2,-3,5), (5,9,t)\]
        is not linearly independent in $\R^3$.
        \begin{solution}
            Let $t = 2$, we have 
            \[ 3(3,1,4) + (-2)(2,-3,5) + (-1)(5,9,2) = (0,0,0),\]
            which means the list is not linearly independent.
        \end{solution}
    \item Verify the assertion in the second bullet point in Example 2.20.
        \begin{solution}
            \textcolor{black}{The list $(2,3,1), (1,-1,2), (7,3,c)$ is linearly dependent in $\F^3$ if and only if $c=8$, as you should verify.}

            If $c = 8$, with the first bullet point we know it is linearly dependent.
            
            Now suppose $c \neq 8$ and it is linearly dependent, 
            let $a_1,a_2,a_3$ be the coeffecients with 
            \[ a_1(2,3,1) + a_2(1,-1,2) + a_3(7,3,c) = 0,\]
            which is same as 
            \[
                \begin{cases}
                    2a_1 + a_2 + 7a_3 &= 0 \\
                    3a_1 - a_2 + 3a_3 &= 0 \\
                    a_1 + 2a_2 + ca_3 &= 0
                \end{cases},
            \]
            the first equation multiple by $7$ add the second equation multiple by $-3$ add the third equation multiple by $-1$, we 
            have 
            \[ (40-5c)a_3 = 0, \]
            since $c \neq 8$, we have $a_3 = 0$, substitue it to the equations above, we have $a_1 = a_2 = 0$, therefore 
            it's linearly independent, which is contradict.
        \end{solution}
    \item \begin{enumerate}[label=(\alph*)]
            \item Show that if we think of $\mC$ as a vector space over $\R$, then the list $(1+i, 1-i)$ is linearly independent.
            \item Show that if we think of $\mC$ as a vector space over $\mC$, then the list $(1+i, 1-i)$ is linearly dependent.
        \end{enumerate}
        \begin{solution}
            \begin{enumerate}[label=(\alph*)]
                \item Suppose $a_1, a_2 \in \R$ with 
                    \[ a_1 (1+i) + a_2 (1-i) = 0, \]
                    it is same with 
                    \[ 
                        \begin{cases}
                            a_1 + a_2 &= 0\\
                            a_1 - a_2 &= 0
                        \end{cases},
                    \]
                    which has only solution 
                    \[ a_1 = 0, \qquad a_2 = 0,\]
                    thus it is linearly independent.
                \item We can find $a_1 = 1+i, a_2 = 1-i \in \mC$, that 
                    \[ a_1 (1+i) + a_2(1-i) = 2i - 2i = 0,\]
                    therefore it is linearly dependent.
            \end{enumerate}
        \end{solution}
    \item Suppose $v_1, v_2, v_3, v_4$ is linearly independent in $V$. Prove that the list 
        \[ v_1-v_2, v_2-v_3,v_3-v_4,v_4\]
        is also linearly independent.
        \begin{solution}
            Assume that $ v_1-v_2, v_2-v_3,v_3-v_4,v_4$ is linearly dependent, there are nonzero $a_1, a_2, a_3, a_4$ with 
            \[ a_1 (v_1-v_2) + a_2(v_2-v_3) + a_3(v_3-v_4) + a_4v_4 = 0,\]
            which is same with
            \[ a_1v_1 + (a_2-a_1)v_2 + (a_3-a_2)v_3 + (a_4-a_3)v_4 = 0,\]
            since $v_1, v_2, v_3, v_4$ is linearly independent, therefore 
            \[ 
                \begin{cases}
                    a_1 &= 0 \\
                    a_2 - a_1 &= 0 \\
                    a_3 - a_2 &= 0 \\
                    a_4 - a_3 &= 0
                \end{cases},
            \]
            which means 
            \[ a_1 = a_2 = a_3 = a_4 = 0 ,\]
            which is a contradict to our assumption (that $a_1, a_2, a_3, a_4$ are nonzero).
        \end{solution}
    \item Prove or give a counterexample: If $v_1, v_2, \ldots, v_m$ is a linearly independent list of vectors in $V$, then 
        \[5v_1-4v_2, v_2,v_3,\ldots, v_m\]
        is linearly independent.
        \begin{solution}
            Assume that it is linearly dependent, therefore there are nonzero $a_1, \ldots, a_m$ that 
            \[ a_1(5v_1-4v_2) + a_2v_2 + \dots + a_mv_m = 0 ,\]
            which can be formed as 
            \[ 5a_1v_1 + (a_2-4a_1)v_2 + a_3v_3 + \dots + a_mv_m = 0,\]
            since $v_1, \ldots, v_m$ is linearly independent, we have 
            \[
                \begin{cases}
                    5a_1 &= 0 \\
                    a_2 - 4a_1 &= 0 \\
                    a_3 &= 0 \\
                    \vdots \\
                    a_m &= 0
                \end{cases},
            \]
            from the first equaiton $5a_1 = 0$ we have $a_1=0$, substitute it to the second equation $a_2-4a_1 = 0$, we have $a_2 = 0$, 
            therefore 
            \[ a_1 = a_2 = \dots = a_m = 0,\]
            which is a contradict to our assumption.

            Thus, this list is linearly independent.
        \end{solution}
    \item Prove or give a counterexample: If $v_1, v_2, \ldots, v_m$ is a linearly independent list of vectors in $v$ and $\lambda \in \F$ with 
        $\lambda \neq 0$, then $\lambda v_1, \lambda v_2, \ldots, \lambda v_m$ is linearly independent.
        \begin{solution}
            Suppose $a_1, \ldots, a_m \in \F$ that 
            \[ a_1\lambda v_1 + \dots + a_m \lambda v_m = 0,\]
            since $v_1, \ldots, v_m$ is linearly independent, means 
            \[a_i\lambda = 0, \qquad i = 1, \ldots, m,\]
            as $\lambda \neq 0$, we have 
            \[ a_i =0, \qquad i=1,\ldots,m,\]
            thus, the list is linearly independent.
        \end{solution}
    \item Prove or give a counterexample: If $v_1, v_2, \ldots, v_m$ and $w_1, \ldots, w_m$ are linearly independent lists of vectors in $\V$,
        then $v_1 + w_1, \ldots, v_m + w_m$ is linearly independent.
        \begin{solution}
            Let 
            \[ v_1 = (1,0), \qquad v_2 = (0,1), \qquad w_1=(0,1), \qquad w_2=(1,0),\]
            it is easy to verify that $v_1, v_2$ is linearly independent and $w_1, w_2$ is linearly independent,
            but $v_1+w_1, v_2+w_2$ which is 
            \[ (1,1), (1,1)\]
            is linearly dependent.
        \end{solution}
    \item Suppose $v_1, \ldots, v_m$ is linearly independent in $V$ and $W \in V$. Prove that if $v_1 + w, \ldots, v_m +w$ is linearly dependent, 
        then $w \in \mspan{v_1,\ldots,v_m}$.
        \begin{solution}
            Since $v_1 + w, \ldots, v_m + w$ are linearly dependent, which means there are nonzero $a_1, \ldots, a_m$ with 
            \[ a_1(v_1+w) + \dots + a_m(v_m + w)  = 0,\]
            let $S = a_1 + \dots + a_m$, the equation above is same with 
            \[ a_1v_1 + \dots + a_m v_m + Sw = 0,\]
            we have $S \neq 0$, or the equation above will be same with 
            \[ a_1 v_1 + \dots + a_mv_m = 0,\]
            and $v_1, \ldots v_m$ is linearly independent, implies 
            \[ a_1=\dots =a_m=0,\]
            which is a contradict with our assumption.
            Since $S \neq 0$, we can formed $w$ as 
            \[ w = \dfrac{-a_1}{S}v_1 + \dots + \dfrac{-a_m}{S}v_m ,\]
            \ie, 
            \[ w \in \mspan{v_1, \ldots, v_m}.\]
        \end{solution}
    \item Suppose $v_1, \ldots, v_m$ is linearly independent in $V$ and $w \in V$. Show that $v_1, \ldots, v_m, w$ is linearly independent if and only if 
        \[ w \notin \mspan{v_1, \ldots, v_m}.\]
        \begin{solution}
            First we suppose that $v_1, \ldots, v_m, w$ is linearly independent and $w \in \mspan{v_1, \ldots, v_m}$,
            therefore 
            \[ w = a_1 v_1 + \dots + a_m v_m ,\]
            \ie, 
            \[ a_1v_1 + dots + a_mv_m - w = 0,\]
            which means $v_1, \ldots, v_m, w$ is linearly dependent, which is contradict.

            Now we suppose $w \notin \mspan{v_1, \ldots, v_m}$ and $v_1, \ldots, v_m, w$ is linearly dependent, therefore there are 
            nonzero $a_1, \ldots, a_m, b$ that 
            \[ a_1 v_1 + \dots + a_mv_m + bw = 0,\]
            we have $b \neq 0$, or the equation above will be 
            \[ a_1v_1 + \dots + a_mv_m = 0,\]
            as $v_1, \ldots, v_m$ is linearly independent, implies that 
            \[ a_1 = \dots = a_m = 0,\]
            which is a contradict, since $b \neq 0$, we have 
            \[ w = \dfrac{-a_1}{b}v_1 + \dots + \dfrac{-a_m}{b}v_m,\]
            which means $w \in \mspan{v_1, \ldots, v_m}$, which is a contradict.
        \end{solution}
    \item Explain why no list of four polynomials spans $\mathcal{P}_4(\F)$.
        \begin{solution}
            We know that the list $1, z, z^2, z^3, z^4$ is linearly independent and they are in $\mathcal{P}_4(\F)$,
            with the fact that 
            \[ \text{\textit{Length of linearly independent list} $\leq$ \textit{Length of span list}}.\]
            We know the length of span list of $\mathcal{P}_4(\F)$ is at least $5$.
        \end{solution}
    \item \hypertarget{2A-13}{Prove} that $V$ is infinite-dimensional if and only if there is a sequence $v_1, v_2, \ldots$ of vectors in $V$ such that $v_1, \ldots, v_m$ is 
        linearly independent for every positive integer $m$.
        \begin{solution}
            Firts we will show that if $V$ is infinite-dimensional, then such a list exists: 
            we can use induction method in $m$,
            \begin{itemize}
                \item If $m=1$, there is non-zero elements in $V$, we choose $v_1 \neq 0 \in V$, therefore $v_1$ is linearly independent;
                \item Now suppose for $m=1,\ldots,k$, we can find $v_1, \ldots, v_k$ is linearly independent;
                \item For $m = k+1$, we can find $v_{k+1} \in V$ which can not be formed as linear combination of $v_1, \ldots, v_k$, 
                    or we have $\mspan{v_1, \ldots, v_k} = V$, which means $V$ is finit-dimensional, is contradict. Since $v_{k+1}$ 
                    can not be formed as linear combination of $v_1, \ldots, v_k$, means $v_1, \ldots, v_k, v_{k+1}$ is linearly independent.
            \end{itemize}

            Now we will show the inversely direction, that if such a list exist, that $V$ is infinite-dimensional:
            As fact it is easy, suppose that $V$ is finite-dimensional and $v_1, \ldots, v_k$ spans $V$, therefore there is 
            not linearly independent list with length greater than $k$, which is contradict to our assumption.
        \end{solution}
    \item Prove that $\F^{\infty}$ is infinite-dimensional.
        \begin{solution}
            
        \end{solution}
    \item Prove that the real vector space of all continuous real-valued functions on the interval $[0,1]$ is infinite-dimensional.
    \item Suppose $p_0,p_1,\ldots,p_m$ are polynomials in $\mathcal{P}_m(\F)$ such that $p_j(2)=0$ for each $j$.
        Prove that $p_0,p_1,\ldots,p_m$ is not linearly independent in $\mathcal{P}_m(\F)$.

\end{enumerate}

\subsection{Bases}

\begin{enumerate}
    \item Find all vector spaces that have exactly one basis.
        \begin{solution}
            The only vector space that has exactly one basis is $\{0\}$ and it is easy to verify. 

            Now suppose there is a vector space $U \neq \{0\}$ that has only one basis, assume the basis is 
            \[ v_1, \ldots, v_m .\]
            it is easy to verify that $v_i \neq 0$, 
            and we will show that 
            \[ v_1 + v_2, v_2, \ldots, v_m\] 
            is also a basis of U:
            let 
            \[ a_1 (v_1+v_2) + a_2v_2 + \dots + a_mv_m = 0,\]
            which is same with 
            \[ a_1v_1 + (a_1+a_2)v_2 + \dots + a_mv_m = 0,\]
            since $v_1, \ldots, v_m$ is basis therefore they are linearly independent, \ie, 
            \[ a_1 = \dots = a_m = 0,\]
            which means $v_1+v_2, v_1, \ldots, v_m$ is linearly independent.

            since $v_1, \ldots, v_m$ spans $U$, means for any $u \in U$, we have 
            \[ u = a_1v_1 + \dots + a_mv_m = a_1(v_1+v_2) + (a_2-a_1)v_2 + \dots + a_mv_m,\]
            means $v_1+v_2, v_2, \ldots, v_m$ spans $U$.

            Thus $v_1+v_2, v_2, \ldots, v_m$ is basis of $U$, but it is different with $v_1, \ldots, v_m$.
        \end{solution}
    \item Verify all the assertions in Example 2.28.
        \begin{solution}
            \begin{enumerate}[label=(\alph*)]
                \item \textcolor{black}{The list $(1,0,\ldots,0),(0,1,0,\ldots,0),(0,\ldots,0,1)$ is a basis of $\F^n$, called the \textbf{standard basis} of $\F^n$.}
                
                    For any $(x_1,\ldots, x_n) \in \F^n$, we can find 
                    \[ (x_1, \ldots, x_n) = x_1(1,0,\ldots) + \dots + x_n(0,\ldots,1),\]
                    therefore these list spans $\F^n$.

                    Suppose the we have 
                    \[ a_1 (1,0,\ldots,0) + \dots + a_n(0,\ldots,1) = (a_1, \ldots, a_n) = 0,\]
                    which means 
                    \[ a_1 = \ldots = a_n = 0,\]
                    therefore this list is linearly independent.
                \item \textcolor{black}{The list $(1,2),(3,5)$ is a basis of $\F^2$.}
                
                    For any $(x,y) \in \F$, we can find $a_1, a_2$ with 
                    \[ a_1(1,2) + a_2(3,5) = (x,y),\]
                    since the equations 
                    \[
                        \begin{cases}
                            a_1 + 3a_2 &= x \\
                            2a_1 + 5a_2 &=y 
                        \end{cases}
                    \]
                    has solution 
                    \[ a_1 = 3y-5x, \qquad a_2=2x-y.\]

                    Let $x=0,y=0$, we have 
                    \[ a_1 = a_2 = 0,\]
                    means this list is linearly independent.

                    Thus this list is a basis of $\F^2$.
                \item \textcolor{black}{The list $(1,2,-4), (7,-5,6)$ is linearly independent in $\F^3$ but is not a basis of $\F^3$ because it does not span $\F^3$.}
                
                    We can prove that this list is linearly independent since the equations
                    \[ 
                        \begin{cases}
                            a_1 + 7a_2 &= 0 \\
                            2a_1 - 5a_2 &= 0\\
                            -4a_1 + 6a_2 &= 0
                        \end{cases}
                    \]
                    has only solution 
                    \[ a_1 = a_2 = a_3 = 0.\]

                \item \textcolor{black}{The list $(1,1,0),(0,0,1)$ is a basis of $\{(x,x,y)\in \F^3: x,y \in \F\}$.}
                
                        It is easy to verigy that the list spans the vector space and it is linearly independent.s
                \item \textcolor{black}{The list $(1,-1,0),(1,0,-1)$ is a basis of \[\{(x,y,z)\in \F^3:x+y+z=0\}.\]}

                        It is easy to verify this list is linearly independent, now we will show that it spans the vector space:
                        since for any $(x,y,z)$ in this vector space, we can find 
                        \[ a_1 = -y,\qquad a_2 = -z\]
                        that 
                        \[ a_1(1,-1,0) + a_2(1,0,-1) = (x,y,z).\]
                \item \textcolor{black}{The list $1, z, \ldots, z^m$ is a basis of $\mathcal{P}_m(\F)$.}
                
                    We have prove that $1,z,\ldots,z^m$ is linearly independent, since it also spans $\mathcal{P}_m(\F)$, therefore it is a basis.
            \end{enumerate}
        \end{solution}
    \item \begin{enumerate}[label=(\alph*)]
            \item Let $U$ be the subspace of $\R^5$ defined by 
                \[ U = \{(x_1,x_2,x_3,x_4,x_5) \in \R^5:x_1=3x_2 \text{ and } x_3=7x_4\}.\]
                Find a basis of $U$.
            \item Extend the basis in part (a) to a basis of $\R^5$.
            \item Find a subspace $W$ of $\R^5$ such that $\R^5 = U \oplus W$.
        \end{enumerate}
        \begin{solution}
            \begin{enumerate}[label=(\alph*)]
                \item The list $(3,1,0,0,0),(0,0,7,1,0), (0,0,0,0,1)$ is a basis of this vector space.
                \item Since $(1,0,0,0,0)$ is not in the span of previous list, we can add $(1,0,0,0,0)$; then we find $(0,0,1,0,0)$ is not in the 
                    span of the previous list, therefore 
                    \[(3,1,0,0,0), (0,0,7,1,0),(0,0,0,0,1),(1,0,0,0,0),(0,0,1,0,0)\]
                    is a basis of $\R^5$.
                \item let $W = \mspan{(1,0,0,0,0), (0,0,1,0,0)}$, we have $U \oplus W = \R^5$.
            \end{enumerate}
        \end{solution}
    \item Prove or disprove: there exists a basis $p_0, p_1, p_2, p_3$ of $\mathcal{P}_3(\F)$ such that none of the polynomials $p_0, p_1, p_2, p_3$ has 
        degree $2$.
        \begin{solution}
            We will show that the polynomial list 
            \[ 1, x, x^3-x^2, x^3, x^4 \]
            satisfies the requirement, since none of the these polynomials has degree $2$, now we will show that it is a basis:
            since the list has length $5$, we only need to show that it is linearly independent: suppose that 
            \[ a_1 + a_2x + a_3(x^3-x^2) + a_4x^3 + a_5x^4 = 0,\]
            observed that only $a_5x^4$ has term $x^4$ implies $a_5 = 0$ also we have 
            \[ a_1 = a_2 =0,\]
            and only $a_3(x^3-x^2)$ has term $x^2$ implies that $a_2=0$, means $a_3 =0$, too, thus 
            it is linearly independent as desired.
        \end{solution}
    \item Suppose $v_1, v_2, v_3, v_4$ is a basis of $V$. Prove that 
        \[ v_1 + v_2, v_2+v_3, v_3+v_4, v_4\]
        is also a basis of $V$.
        \begin{solution}
            We have 
            \[ \dim{V} = 4,\]
            means we only need to show that $v_1 + v_2, v_2+v_3, v_3+v_4, v_4$ is linearly independent, then it is a basis of $V$ since it has the right length.
            suppose that 
            \[ a_1(v_1+v_2) + a_2(v_2 + v_3) + a_3(v_3+v_4) + a_4v_4 = 0,\]
            which can be formed as 
            \[ a_1v_1 + (a_1 + a_2)v_2 + (a_2 + a_3)v_3 + (a_3 + a_4)v_4 = 0,\]
            since $v_1, v_2, v_3, v_4$ is linearly independent, then we have 
            \[ a_1 = 0, \ a_1 + a_2 =0, \ a_2 + a_3 = 0, \ a_3 + a_4 = 0,\]
            which implies 
            \[ a_1 = a_2 = a_3 = a_4 = 0,\]
            as we desired.
        \end{solution}
    \item Prove or give a counterexample: If $v_1, v_2, v_3, v_4$ is a basis of $V$ and $U$ is a subspace of $V$ such that 
        $v_1,v_2 \in U$ and $v_3 \notin U$ and $v_4 \notin U$, then $v_1, v_2$ is a basis of $U$.
        \begin{solution}
            Consider $U = \mspan{v_1, v_2, v_3 + v_4}$, we will show that $U$ satisfies the conditions above, but $v_1,v_2$ is not the basis of $U$.

            It is obviousely that $v_1,v_2 \in U$, soppuse that $v_3 \in U$, we have $v_4 \in U$, means there is a linearly independent list with length $4$, 
            \ie, $v_1, v_2, v_3, v_4$ but the span list has length $3$, which is imposible, therefore 
            \[ v_3 \notin U,\]
            with same method, we have 
            \[ v_4 \notin U,\]
            thus $U$ satisfies the condition, and we know that the element $v_3 + v_4$ can not be formed as linear combination of $v_1$ and $v_2$, therefore 
            $v_1, v_2$ is not a basis of $U$.
        \end{solution}
    \item Suppose $U$ and $W$ are subspaces of $V$ such that $V = U \oplus W$. Suppose also that $u_1, \ldots, u_m$ is a basis of $U$ 
        and $w_1, \ldots, w_n$ is a basis of $W$. Prove that 
        \[ u_1, \ldots,u_m, w_1, \ldots, w_n\]
        is a basis of $V$.
        \begin{solution}
            First we will show that 
            \[ V = \mspan{u_1, \ldots, u_m, w_1, \ldots, w_n},\]
            since $V = U \oplus W$, we suppose that $p \in V$, therefore $p$ can be formed as 
            \[ p = u + w, \qquad u \in U, w \in W,\]
            and with $u_1, \ldots u_m$ is basis of $U$ and $u$ is an element of $U$, we have 
            \[ u = a_1 u_1 + \dots + a_mu_m,\]
            and simily we have 
            \[ w = b_1w_1 + \dots + b_nw_n,\]
            therefore 
            \[ p = a_1u_1 + \dots + a_mu_m + b_1w_1 + \dots + b_nw_n,\]
            means 
            \[ p \in \mspan{u_1, \ldots, u_m, w_1, \ldots, w_n}.\]
            Then we consider $p$ as an element of the span of list $u_1, \ldots, u_m, w_1, \ldots, w_n$, \ie, 
            \[ p = a_1u_1 + \dots + a_mu_m + b_1w_1 + \dots + b_nw_n ,\]
            since $u_1, \ldots, u_m$ is basis of $U$, we have 
            \[ a_1u_1 + \dots + a_mu_m \in U, \]
            let it to be $u$, \ie, 
            \[ u = a_1u_1 + \dots + a_mu_m \in U,\]
            similarly we have 
            \[ w = b_1w_1 + \dots + b_nw_n \in W,\]
            therefore 
            \[ p \in U + W = V,\]
            thus we have 
            \[ V = \mspan{u_1, \ldots, u_m, w_1, \ldots, w_n}.\]

            Since $U+W$ is a direct sum and therefore $u_1, \ldots, u_m, w_1, \ldots, w_n$ is linearly independent, implies that 
            \[ u_1, \ldots, u_m, w_1, \ldots, w_n\]
            is a basis of $V$.
        \end{solution}
\end{enumerate}

\subsection{Dimension}
\begin{enumerate}
    \item Suppose $V$ is finite-dimensional and $U$ is a subspace of $V$ such that $\dim{U} = \dim{V}$. Prove that $U = V$.
    \item Show that the subspaces of $\R^2$ are precisely $\{0\}$, $\R^2$, and all lines in $\R^2$ through the origin.
    \item Show that the subspaces of $\R^3$ are precisely $\{0\}$, $\R^3$, and all lines in $\R^3$ through the origin, and all
        planes in $\R^3$ through the origin.
    \item \begin{enumerate}[label=(\alph*)]
            \item Let $U = \{p \in \mathcal{P}_4(\F): p(6)=0.\}$. Find a basis of $U$.
            \item Extend the basis in part (a) to a basis of $\mathcal{P}_4(\F)$.
            \item Find a subspace $W$ of $\mathcal{P}_4(\F)$ such that $\mathcal{P}_4(\F) = U \oplus W$.
        \end{enumerate}
    \item \begin{enumerate}[label=(\alph*)]
            \item Let $U = \{p \in \mathcal{P}_4(\R):p''(6)=0\}$.Find a basis of $U$.
            \item Extend the basis in part (a) to a basis of $\mathcal{P}_4(\R)$.
            \item Find a subspace $W$ of $\mathcal{P}_4(\R)$ such that $\mathcal{P}_4(\R) = U \oplus W$.
        \end{enumerate}
    \item \begin{enumerate}[label=(\alph*)]
            \item Let $U = \{p \in \mathcal{P}_4(\R):p(2)=p(5)=p(6)\}$.Find a basis of $U$.
            \item Extend the basis in part (a) to a basis of $\mathcal{P}_4(\R)$.
            \item Find a subspace $W$ of $\mathcal{P}_4(\R)$ such that $\mathcal{P}_4(\R) = U \oplus W$.
        \end{enumerate}
    \item \begin{enumerate}[label=(\alph*)]
            \item Let $U = \{p \in \mathcal{P}_4(\R):\int_{-1}^{1} p =0\}$.Find a basis of $U$.
            \item Extend the basis in part (a) to a basis of $\mathcal{P}_4(\R)$.
            \item Find a subspace $W$ of $\mathcal{P}_4(\R)$ such that $\mathcal{P}_4(\R) = U \oplus W$.
        \end{enumerate}
    \item Suppose $v_1,\ldots,v_m$ is linearly independent in $V$ and $w \in V$. Prove that 
        \[ \dim{\mspan{v_1+w,\ldots,v_m+w}} \geq m - 1.\]
    \item Suppose $p_0,p_1,\ldots,p_m \in \mathcal{P}(\F)$ are such that each $p_j$ has degree $j$. Prove that 
        $p_0,p_1,\ldots,p_m$ is a basis of $\mathcal{P}_m(\F)$.
    \item Suppose that $U$ and $W$ are subspaces of $\R^8$ such that $\dim{U} = 3$, $\dim{W} = 5$, and 
        $U + W = \R^8$. Prove that $\R^8 = U \oplus W$.
    \item Suppose $U$ and $W$ are both five-dimensional subspaces of $\R^9$. Prove that $U \cap W \neq \{0\}$.
    \item Suppose $U$ and $W$ are both $4$-dimensional subspaces of $\mC^6$. Prove that there exist two vectors 
        in $U \cap W$ such that neigher of these vectors is a scalar multiple of the other.
    \item Suppose $U_1, \ldots, U_m$ are finite-dimensional subspaces of $V$. Prove that 
        $U_1 + \dots + U_m$ is finite-dimensional and 
        \[ \dim{(U_1 + \dots + U_m)} \leq \dim{U_1} + \dots + \dim{U_m}. \]
        \begin{solution}
            First We will show that \textbf{the sum of two finite-dimensional vector space is a finite-dimensional.}

            Suppose $U$ and $V$ are these two finite-dimensional vector space, therefore they have basis, since there are span list of them and
            span list include basis, suppose the that $u_1, \ldots, u_n$ and $v_1, \ldots, v_m$ are basis of $U$ and $V$, respectively. Now We will 
            show that 
            \[ u_1, \ldots, u_n, v_1, \ldots, v_m \]
            spans $U+V$: for any element $p \in U+V$, we have $p = u+v, u \in U, v \in V$, and we have 
            \[ u = a_1u_1 + \dots + a_nu_n, \qquad v = b_1v_1 + \dots + b_mv_m ,\]
            therefore 
            \[ p = a_1u_1 + \dots + a_nu_n + b_1v_1 + \dots + b_mv_m \]
            which is an element of $\mspan{u_1, \ldots, u_n, v_1, \ldots, v_m}$, and we can prove that for any element 
            in the $\mspan{u_1, \ldots, u_n, v_1, \ldots, v_m}$, it is also a element of $U + v$, therefore 
            \[ U+V = \mspan{u_1, \ldots, u_n, v_1, \ldots, v_m},\]
            \ie, $U +V$ is finite-dimensional.

            Use the inductive method, we can prove that $U_1 + \dots + U_m$ is finite-dimensional.

            In order to prove 
            \[ \dim{(U_1 + \dots + U_m)} \leq \dim{U_1} + \dots + \dim{U_m},\]
            we show that for any finite-dimensional vector space $U$ and $V$, we have 
            \[ \dim{(U+V)} = \dim{U} + \dim{V} - \dim{(U \cap V)} \leq \dim{U} + \dim{V}\]
            since $\dim{U \cap V} \geq 0$, therefore 
            \begin{align*}
                \dim{(U_1 + \dots + U_m)} &\leq \dim{U_1} + \dim{(U_2 + \dots + U_m)} \\
                                          &\leq \dots \\
                                          &\leq \dim{U_1} + \dots + \dim{U_m}.
            \end{align*}
        \end{solution}
    \item Suppose $V$ is finite-dimensional, with $\dim{V} = n \geq 1$. Prove that there exist $1$-dimensional subspaces $U_1, \ldots, U_n$ of 
        $V$ such that 
        \[ V = U_1 \oplus \dots \oplus U_n. \]
        \begin{solution}
            In order to show this, we prove another satetement: \textbf{For give $V$, we can find $1$-dimensional subsapce $U$ and $(n-1)$-dimensional $W$, that}
            \[ V = U \oplus W.\]
            We suppose that list 
            \[ v_1, \ldots, v_n\]
            is basis of $V$, let $U = \mspan{v_1}$ and $W = \mspan{v_2, \ldots,v_m}$, From 2.34 we know that 
            \[ V = U \oplus W,\]
            continue this method, we can prove that 
            \[ V = \mspan{v_1} + \dots + \mspan{v_n} .\]
        \end{solution}
    \item Suppose $U_1, \ldots, U_m$ are finite-dimensional subspaces of $V$ such that $U_1 + \dots + U_m$ is direct sum. 
        Prove that $U_1 \oplus \dots \oplus U_m$ is finite-dimensional and 
        \[ \dim{U_1 \oplus \dots \oplus U_m} = \dim{U_1} + \dots + \dim{U_m}. \]
        \begin{solution}
            If we can prove that when $m = 2$, the statement hold, then then statement hold for any positive interge $m$.

            Suppose that $u_1,\ldots,u_n$ and $v_1,\ldots, v_m$ are the basis of $U_1$ and $U_2$, respectively, then 
            \[ u_1, \ldots, u_n, v_1, \ldots, v_m\]
            spans $U_1 + U_2$, which means $U+V$ is finite-dimensional.
            
            Since $U_1 + U_2 $ is direct sum, we have $U_1 \cap U_2 = \{0\}$, \ie, $\dim{(U_1 \cap U_2)} = 0$ therefore 
            \[ \dim{(U_1 \oplus U_2)} = \dim{U_1} + \dim{U_2}.\]
        \end{solution}
    \item You might guess, by analogy with the formula for the number of elements in the union of three subsets of a finite set, 
        that if $U_1, U_2, U_3$ are subspaces of a finite-dimensional vector space, then 
        \begin{align*}
            \dim{(U_1 + U_2 + U_3)} =& \dim{U_1} + \dim{U_2} + \dim{U_3} \\
                                   & - \dim({U_1 \cap U_2)} - \dim{(U_1 \cap U_3)} - \dim{(U_2 \cap U_3)} \\
                                   & + \dim{(U_1 \cap U_2 \cap U3)}.
        \end{align*}
        Prove this or give a counterexample.
        \begin{solution}
            We will give a counterexample:
            Consider that 
            \[ U_1=\{(x,x)\in \R^2:x\in \R\},\ U_2=\{(x,0)\in \R^2: x\in \R\}, \ U_3=\{(0,x)\in \R^2: x\in \R\},\]
            we have 
            \[ \dim{U_1} = \dim{U_2} = \dim{U_3} = 1,\]
            and since $U_1 + U_2 + U_3 = \R^2$, we have 
            \[ \dim{(U_1 + U_2 + U_3)} = 2,\]
            and 
            \[ U_1 \cap U_2 = U_1 \cap U_3 = U_2 \cap U_3 = U_1 \cap U_2 \cap U_3 = \{0\},\]
            we have 
            \[ \dim{(U_1\cap U_2)} = \dim{(U_1 \cap U_3)} = \dim{(U_2 \cap U_3)} = \dim{(U_1 \cap U_2 \cap U_3)} = 0,\]
            therefore the left-hand side of the equation is $2$, the right-hand side is $2$, since 
            \[ 3 \neq 2,\]
            we conclude that the equation does not hold all the time.
        \end{solution}
\end{enumerate}